\documentclass[12pt,]{book}
\usepackage{lmodern}
\usepackage{setspace}
\setstretch{1.5}
\usepackage{amssymb,amsmath}
\usepackage{ifxetex,ifluatex}
\usepackage{fixltx2e} % provides \textsubscript
\ifnum 0\ifxetex 1\fi\ifluatex 1\fi=0 % if pdftex
  \usepackage[T1]{fontenc}
  \usepackage[utf8]{inputenc}
\else % if luatex or xelatex
  \ifxetex
    \usepackage{mathspec}
  \else
    \usepackage{fontspec}
  \fi
  \defaultfontfeatures{Ligatures=TeX,Scale=MatchLowercase}
\fi
% use upquote if available, for straight quotes in verbatim environments
\IfFileExists{upquote.sty}{\usepackage{upquote}}{}
% use microtype if available
\IfFileExists{microtype.sty}{%
\usepackage{microtype}
\UseMicrotypeSet[protrusion]{basicmath} % disable protrusion for tt fonts
}{}
\usepackage[margin=1in]{geometry}
\usepackage{hyperref}
\hypersetup{unicode=true,
            pdftitle={Lab Handbook},
            pdfauthor={Candice Morey},
            pdfborder={0 0 0},
            breaklinks=true}
\urlstyle{same}  % don't use monospace font for urls
\usepackage{natbib}
\bibliographystyle{apalike}
\usepackage{color}
\usepackage{fancyvrb}
\newcommand{\VerbBar}{|}
\newcommand{\VERB}{\Verb[commandchars=\\\{\}]}
\DefineVerbatimEnvironment{Highlighting}{Verbatim}{commandchars=\\\{\}}
% Add ',fontsize=\small' for more characters per line
\usepackage{framed}
\definecolor{shadecolor}{RGB}{248,248,248}
\newenvironment{Shaded}{\begin{snugshade}}{\end{snugshade}}
\newcommand{\KeywordTok}[1]{\textcolor[rgb]{0.13,0.29,0.53}{\textbf{#1}}}
\newcommand{\DataTypeTok}[1]{\textcolor[rgb]{0.13,0.29,0.53}{#1}}
\newcommand{\DecValTok}[1]{\textcolor[rgb]{0.00,0.00,0.81}{#1}}
\newcommand{\BaseNTok}[1]{\textcolor[rgb]{0.00,0.00,0.81}{#1}}
\newcommand{\FloatTok}[1]{\textcolor[rgb]{0.00,0.00,0.81}{#1}}
\newcommand{\ConstantTok}[1]{\textcolor[rgb]{0.00,0.00,0.00}{#1}}
\newcommand{\CharTok}[1]{\textcolor[rgb]{0.31,0.60,0.02}{#1}}
\newcommand{\SpecialCharTok}[1]{\textcolor[rgb]{0.00,0.00,0.00}{#1}}
\newcommand{\StringTok}[1]{\textcolor[rgb]{0.31,0.60,0.02}{#1}}
\newcommand{\VerbatimStringTok}[1]{\textcolor[rgb]{0.31,0.60,0.02}{#1}}
\newcommand{\SpecialStringTok}[1]{\textcolor[rgb]{0.31,0.60,0.02}{#1}}
\newcommand{\ImportTok}[1]{#1}
\newcommand{\CommentTok}[1]{\textcolor[rgb]{0.56,0.35,0.01}{\textit{#1}}}
\newcommand{\DocumentationTok}[1]{\textcolor[rgb]{0.56,0.35,0.01}{\textbf{\textit{#1}}}}
\newcommand{\AnnotationTok}[1]{\textcolor[rgb]{0.56,0.35,0.01}{\textbf{\textit{#1}}}}
\newcommand{\CommentVarTok}[1]{\textcolor[rgb]{0.56,0.35,0.01}{\textbf{\textit{#1}}}}
\newcommand{\OtherTok}[1]{\textcolor[rgb]{0.56,0.35,0.01}{#1}}
\newcommand{\FunctionTok}[1]{\textcolor[rgb]{0.00,0.00,0.00}{#1}}
\newcommand{\VariableTok}[1]{\textcolor[rgb]{0.00,0.00,0.00}{#1}}
\newcommand{\ControlFlowTok}[1]{\textcolor[rgb]{0.13,0.29,0.53}{\textbf{#1}}}
\newcommand{\OperatorTok}[1]{\textcolor[rgb]{0.81,0.36,0.00}{\textbf{#1}}}
\newcommand{\BuiltInTok}[1]{#1}
\newcommand{\ExtensionTok}[1]{#1}
\newcommand{\PreprocessorTok}[1]{\textcolor[rgb]{0.56,0.35,0.01}{\textit{#1}}}
\newcommand{\AttributeTok}[1]{\textcolor[rgb]{0.77,0.63,0.00}{#1}}
\newcommand{\RegionMarkerTok}[1]{#1}
\newcommand{\InformationTok}[1]{\textcolor[rgb]{0.56,0.35,0.01}{\textbf{\textit{#1}}}}
\newcommand{\WarningTok}[1]{\textcolor[rgb]{0.56,0.35,0.01}{\textbf{\textit{#1}}}}
\newcommand{\AlertTok}[1]{\textcolor[rgb]{0.94,0.16,0.16}{#1}}
\newcommand{\ErrorTok}[1]{\textcolor[rgb]{0.64,0.00,0.00}{\textbf{#1}}}
\newcommand{\NormalTok}[1]{#1}
\usepackage{longtable,booktabs}
\usepackage{graphicx,grffile}
\makeatletter
\def\maxwidth{\ifdim\Gin@nat@width>\linewidth\linewidth\else\Gin@nat@width\fi}
\def\maxheight{\ifdim\Gin@nat@height>\textheight\textheight\else\Gin@nat@height\fi}
\makeatother
% Scale images if necessary, so that they will not overflow the page
% margins by default, and it is still possible to overwrite the defaults
% using explicit options in \includegraphics[width, height, ...]{}
\setkeys{Gin}{width=\maxwidth,height=\maxheight,keepaspectratio}
\IfFileExists{parskip.sty}{%
\usepackage{parskip}
}{% else
\setlength{\parindent}{0pt}
\setlength{\parskip}{6pt plus 2pt minus 1pt}
}
\setlength{\emergencystretch}{3em}  % prevent overfull lines
\providecommand{\tightlist}{%
  \setlength{\itemsep}{0pt}\setlength{\parskip}{0pt}}
\setcounter{secnumdepth}{5}
% Redefines (sub)paragraphs to behave more like sections
\ifx\paragraph\undefined\else
\let\oldparagraph\paragraph
\renewcommand{\paragraph}[1]{\oldparagraph{#1}\mbox{}}
\fi
\ifx\subparagraph\undefined\else
\let\oldsubparagraph\subparagraph
\renewcommand{\subparagraph}[1]{\oldsubparagraph{#1}\mbox{}}
\fi

%%% Use protect on footnotes to avoid problems with footnotes in titles
\let\rmarkdownfootnote\footnote%
\def\footnote{\protect\rmarkdownfootnote}

%%% Change title format to be more compact
\usepackage{titling}

% Create subtitle command for use in maketitle
\newcommand{\subtitle}[1]{
  \posttitle{
    \begin{center}\large#1\end{center}
    }
}

\setlength{\droptitle}{-2em}

  \title{Lab Handbook}
    \pretitle{\vspace{\droptitle}\centering\huge}
  \posttitle{\par}
    \author{Candice Morey}
    \preauthor{\centering\large\emph}
  \postauthor{\par}
      \predate{\centering\large\emph}
  \postdate{\par}
    \date{2018-10-27}

\usepackage{booktabs}
\usepackage{amsthm}
\makeatletter
\def\thm@space@setup{%
  \thm@preskip=8pt plus 2pt minus 4pt
  \thm@postskip=\thm@preskip
}
\makeatother

\usepackage{amsthm}
\newtheorem{theorem}{Theorem}[chapter]
\newtheorem{lemma}{Lemma}[chapter]
\theoremstyle{definition}
\newtheorem{definition}{Definition}[chapter]
\newtheorem{corollary}{Corollary}[chapter]
\newtheorem{proposition}{Proposition}[chapter]
\theoremstyle{definition}
\newtheorem{example}{Example}[chapter]
\theoremstyle{definition}
\newtheorem{exercise}{Exercise}[chapter]
\theoremstyle{remark}
\newtheorem*{remark}{Remark}
\newtheorem*{solution}{Solution}
\begin{document}
\maketitle

{
\setcounter{tocdepth}{1}
\tableofcontents
}
\chapter{Joining a lab}\label{joining-a-lab}

\begin{quote}
No man is an island,

Entire of itself,

Every man is a piece of the continent,

A part of the main.

--- John Donne
\end{quote}

Although you will be familiar with the names of a handful of scientific
heroes, science does not actually advance from the rapid insights of
rare geniuses. Scientific knowledge accumulates through the consistent,
often painstaking, efforts of groups of people. Across the world, webs
of laboratories focusing on related topics work toward the common goal
of understanding human memory and communicating how best to use our
emerging knowledge to improve peoples' lives. You have opened this
manual because you have joined one such group. The purpose of this
manual is to help you understand your role in this endeavor and how to
contribute in a way that makes your work maximally useful both to your
local colleagues, your international colleagues, and the public, both
now and branching into the future.

Joining a lab is about taking what you have to offer - your intellect,
your skills, your diligence - and transforming them into new
information. This is not just about you working for the lab: the lab
also helps you acquire new skills, start your work from a more advanced
point than you would be capable of on your own, and eventually boost the
signal for transmitting the knowledge you helped create. Everyone
contributing to the lab's functioning - undergraduate researchers just
starting out, early career researchers, university support staff, the
principal investigator - plays a vital role in this effort. To maximize
the utility of the work we all do, it is crucial for everyone to:

\begin{enumerate}
\def\labelenumi{\arabic{enumi}.}
\item
  Know where to find information about the lab and the procedures we
  follow
\item
  Use the common conventions and procedures we have established
\item
  Propose and implement improvements to this workflow as new, better
  methods become available
\item
  Communicate clearly and openly about our work
\end{enumerate}

Lab procedures have been designed to ensure that the effort you put into
your lab work goes as far as possible. Implementing these procedures may
require learning something you don't currently know how to do, or doing
a task a bit differently than you have done in another lab. We have
examples and support available to help you manage this and learn
quickly. No one should need to re-invent anything in this lab that can
be copied and modified. You will not be starting your work from scratch,
but with at least some building blocks that other researchers have
prepared before.

\section{Hierarchy}\label{hierarchy}

\subsection{Our core team}\label{our-core-team}

Labs naturally have a hierarchy. While there is no unimportant work in
the lab (it's all important), there is some work that requires a mature
knowledge of the literature and a bird's eye view of our international
colleagues' latest work. The \textbf{\emph{PI}} (for Principal
Investigator) and more senior research assistants are more likely to
have this perspective. The most senior staff are also the most
permanent, and will know the most about what we have done before, what
other labs looking at similar topics are doing right now, and what the
most glaring gaps in our knowledge about human memory are. The PI in
particular will be the person most likely to be able to see how all the
projects in the lab relate to each other and to other contemporary
research, to see opportunities for publishing individual pieces of
research or combining apparently different strains of research into a
single article. The PI's ability to do this well depends on everyone's
good documentation of their work and open communication about it.
Ensuring that the state of your project is always discoverable to the PI
increases the chances that it can form part of an important
communication, which benefits everyone involved.

\textbf{\emph{Post-docs}} (for post-doctoral researchers) have already
earned their PhD, but have not yet started their own lab. Post-docs work
here in order to learn more about working memory, get a bit more
research experience and more publications, or learn about a method that
they did not have the chance to learn about during their post-graduate
studies. We are currently trying to secure funding to offer
post-doctoral researchers the chance to work on particular projects, but
we would also be interested in hosting post-doctoral researchers who
want to apply for their own funding to work at Cardiff University on
something related to working memory. Interested people should contact
Candice Morey
(\href{mailto:MoreyC@cardiff.ac.uk}{\nolinkurl{MoreyC@cardiff.ac.uk}})
to discuss ideas for proposals.

Post-docs are the most senior members of the lab after the PI. Depending
on their mission, they might be working on their own project related to
the lab's aims, or they might be leading a project that the PI secured
funding for. They have experience working in other labs, and can bring
their knowledge to our procedures and help us improve them.

\textbf{\emph{Postgraduate Researchers}} (PGRs) are undertaking lab work
as part of their preparation of a postgraduate thesis. PGRs are usually
PhD students working for 3 years, or maybe 1 + 3 years if their funding
also includes a year of MSc study. PGRs' main priority is completing a
coherent body of research for their thesis. All of the research they
undertake for that project could end up being published on its own, but
pieces could also be integrated into other lab projects. This is one
reason cross-project continuity is important. PGRs will usually move on
to a research job elsewhere once their education is complete. At that
point, research they carried out that hasn't been published yet may not
be their top priority, and the PI and post-docs can be helpful in
ensuring that unfinished strands of work continue to flourish.

PGRs may also be paid researchers carrying out designated, funded
laboratory work. For most intents, temporary paid PGRs play similar
roles to PhD students, but paid PGRs will be expected to work on
multiple projects, as hours permit. Paid PGRs will have the chance to be
involved deeply in projects, and may have the opportunity to take
ownership of a line of research. Unlike PhD students, paid PGR time may
be re-directed by the PI to deal with urgent work, like collecting data
needed for an invited revision.

\textbf{\emph{Undergraduate Researchers}} (UGRs) are a diverse group of
students with a variety of reasons for joining the lab. Some will join
the lab to carry out their final-year project, and will be active lab
members from 1 October until 30 April of a single academic year. Others
might be visiting students (on Erasmus fellowships or an internship
scheme) working for a set period as long as a calendar year. Others
might be working for a short period during the summer (e.g., on a CUROP
fellowship). Most UGRs are involved for less than a year, and therefore
must be integrated into a larger project. UGRs learn basic research
skills working in the lab, and if involved in a self-contained project
(like a UG dissertation), they also take responsibility for interim
communications about the work they carried out. Because their time in
the lab is so limited, it is very important that UG research is
well-documented so that the lab can efficiently carry it forward.

\subsection{How the hierarchy works}\label{how-the-hierarchy-works}

The PI, for better or worse, shoulders responsibility for the work
conducted by her lab group. While everyone involved in the work will be
acknowledged when work we have done is published or praised, the PI will
always be primarily responsible for correcting problems when they arise,
no matter who really caused them. Our work can be questioned years after
it has been carried out and published, meaning the PI is the only person
committed to this for long enough to realistically keep this
committment.

For some post-doctoral and PGR projects, the researchers involved might
share a long-term commitment to the research and be the ``local PI'' on
the work. In those cases, they will act as the primary person
responsible for those projects. Even so, the PI must always have access
to enough information about these projects to independently reproduce
analyses and replicate findings.

While the PI thinks in terms of large, multi-experiment projects, lab
researchers at all levels will have the responsibility for individual
experiments, projects, or component projects. Elements of any project
must always be documented. Every project has designated milestones at
which documentation should be completed, backed-up, and shared (at least
with the lab group, often publicly). This standard workflow will be
described in more depth later. Highlights:

\begin{enumerate}
\def\labelenumi{\arabic{enumi}.}
\item
  Whenever a project is communicated at a conference or published, the
  documentation, anonymized data, and reproducible analysis scripts
  supporting the communication must go live unless doing so would
  violate our ethics agreement (which in our lab, should not ordinarily
  happen).
\item
  Whenever a lab member moves on from the lab, every project s/he led
  must be documented and made accessible to the PI on
  \textbf{\emph{OSF}} (for Open Science Framework; osf.io). At the
  point, if the project is not published, the PI must be given full
  editing authority along with the former lab member.
\end{enumerate}

The work we do in the lab is the PI's life's work. You can count on her
to see where the projects you did that were not publishable on their own
may become useful when combined with the output of other projects. In
our list of publications, the contributions of UGR and PGR lab members
are emphasized. Data sufficient for a research paper can take years to
accumulate; in some of these cases, the student co-authors had completed
their piece of the project many years before the paper was published.
The ongoing work of our lab can help boost your work even after you have
moved on, but that is only possible if the remaining lab members can
fully understand and reproduce the work you did.

No one can accomplish the work we are striving to do alone. By joining
this lab, you are jumping into a big, partially-completed project,
whether you realize it or not. Help me make your efforts flourish for
this big endeavor and for your career by learning about how this one cog
in the grand machinery of cognitive science operates.

\section{Current lab personnel}\label{current-lab-personnel}

Dr.~Candice Coker Morey, Prinicipal Investigator

Dr.~Tanya Joseph, Post-doctoral Research Assistant

\subsection{PGRs}\label{pgrs}

\subsection{UGRs}\label{ugrs}

Laura Bentley

Emily Charnaud

Kirsten McDermott

Jone McKeever Garay

Eden Rendell

\section{School of Psychology staff who support
us}\label{school-of-psychology-staff-who-support-us}

\begin{itemize}
\item
  IT services
  (\href{mailto:it-servicedesk@cardiff.ac.uk}{\nolinkurl{it-servicedesk@cardiff.ac.uk}})
  help with computing equipment needs.
\item
  Research administrators
  (\href{mailto:psych-research-admin@cardiff.ac.uk}{\nolinkurl{psych-research-admin@cardiff.ac.uk}})
  help with putting together and submitting grant applications.
\item
  Open Access adminstrators need to know when a paper has been accepted
  for publication, and can help facilitate making it OA and paying OA
  journal fees (Lisa Kennedy,
  \href{mailto:Psych-OpenAccess@cardiff.ac.uk}{\nolinkurl{Psych-OpenAccess@cardiff.ac.uk}})
\item
  Open Science Working Group (OSWG; Candice Morey and Chris Chambers,
  Chairs) offers advice on working transparently and shapes the School
  of Psychology's policy on research transparency
\end{itemize}

\section{Local collabortors and academics interested in memory or
related
things}\label{local-collabortors-and-academics-interested-in-memory-or-related-things}

\href{http://psych.cf.ac.uk/contactsandpeople/jonesdm.php}{Prof.~Dylan
Jones}

\href{http://psych.cf.ac.uk/contactsandpeople/macken.php}{Prof.~Bill
Macken}

\href{http://psych.cf.ac.uk/contactsandpeople/moreyr.php}{Dr.~Richard
Morey}

\href{http://psych.cf.ac.uk/contactsandpeople/morganphil.php}{Dr.~Phil
Morgan}

\href{http://psych.cf.ac.uk/contactsandpeople/evanslh.php}{Dr.~Lisa
Evans}

\section{Current and recent international
collaborators}\label{current-and-recent-international-collaborators}

\href{https://medhealth.leeds.ac.uk/profile/1300/958/richard_allen}{Dr.~Richard
Allen, Leeds University}

\href{https://www.boystownhospital.org/research/Faculty/Pages/Angela-AuBuchon.aspx}{Dr.~Angela
AuBuchon, Boys Town National Research Hospital}

\href{https://www.ed.ac.uk/profile/bonnie-auyeung}{Dr.~Bonnie Auyeung,
University of Edinburgh}

\href{https://www.strath.ac.uk/staff/brownlouisedr/}{Dr.~Louise Brown,
University of Strathclyde}

\href{https://www.ed.ac.uk/profile/nicolas-chevalier}{Dr.~Nic Chevalier,
University of Edinburgh}

\href{http://memory.psych.missouri.edu/cowan.shtml}{Prof.~Nelson Cowan,
University of Missouri}

\href{https://www.qmu.ac.uk/schools-and-divisions/psychology-and-sociology/psychology-and-sociology-staff/dr-stephen-darling/}{Dr.~Steve
Darling, Queen Margaret University}

\href{https://www.lsu.edu/hss/psychology/faculty/cognitive/elliott.php}{Prof.~Emily
Elliott, Louisiana State University}

\href{https://medhealth.leeds.ac.uk/profile/1300/956/jelena_havelka}{Dr.~Jelena
Havelka, Leeds University}

\href{https://www.uni-koblenz-landau.de/en/campus-landau/fb8/entwicklungspsychologie-und-padagogische-psychologie-en/team/prof-julia-karbach}{Prof.~Julia
Karbach, University of Koblenz-Landau}

\href{https://www.unige.ch/fapse/decopsy/group/patrick/}{Dr.~Naomi
Langerock, University of Geneva}

\href{https://psychology.missouri.edu/people/rhodes}{Dr.~Stephen Rhodes,
University of Missouri}

\href{https://www.unige.ch/fapse/decopsy/group/evie/}{Dr.~Evie Vergauwe,
University of Geneva}

\href{https://www.sheffield.ac.uk/psychology/staff/academic/claudia-von_bastian}{Dr.~Claudia
von Bastian, Sheffield University}

\section{External PGRs}\label{external-pgrs}

Erminia Fiorentino, University of Edinburgh

Anthea Allan, Queen Margaret University

\section{Alumni}\label{alumni}

Dr Jason Doherty (PGR co-supervisee, University of Edinburgh)

Dr.~Jaroslaw Lelonkiewcisz (PGR, University of Edinburgh)

Dr.~Jonathan Mall (PGR supervisee, Rijksuniversiteit Groningen)

Dr.~Edyta Sasin (PGR co-supervisee, Rijksuniversiteit Groningen)

Dr.~Florian Sense (PGR co-supervisee \& UGR project, Rijksuniversiteit
Groningen)

Anthea Allan (MSc project, University of Edinburgh)

Margot Holweg (MSc project, Rijksuniversiteit Groningen)

Monica Miron (MSc project, University of Edinburgh)

Lewis Montgomery (MSc project, University of Edinburgh)

Rob Nijenkamp (MSc project, Rijksuniversiteit Groningen)

Marina Nikoletsopoulou (MSc project, University of Edinburgh)

Lieke Roetmann (MSc project, University of Edinburgh)

Karim Rivera-Lares (MSc project, University of Edinburgh)

Madeleine van der Reijden (MSc project, Rijksuniversiteit Groningen)

Gary Wells (MSc project, University of Edinburgh)

Andrea Winkens (MSc project, Rijksuniversiteit Groningen)

Ieva Zeromskaite (MSc project, University of Edinburgh)

Amy Andrews (UGR, University of Edinburgh)

Libby Best (UGR project, Cardiff University)

Dr.~Malte Bieler (UGR project, Rijksuniversiteit Groningen)

Rieke Buggenthin (UGR project, Rijksuniversiteit Gronginen)

Yongqui Cong (UGR project, Rijksuniversiteit Groningen)

Gita Gudka (UGR project, Cardiff University)

Christian Hummeluhr (UGR, Rijksuniversiteit Groningen)

Will King (UGR project, University of Edinburgh)

Franziska Lehnert (UGR project, Rijksuniversiteit Groningen)

Tuomos Ludstrom (UGR, Rijksuniversiteit Groningen)

Robin Macy (UGR project, University of Edinburgh)

Silvana Mareva (UGR, University of Edinburgh)

Leire Martin Mendez (UGR, Rijksuniversiteit Groningen)

Anne Muth (UGR project, Rijksuniversiteit Groningen)

Sophia Oelmann (UGR project, Rijksuniversiteit Groningen)

Viktorija Pratuseviciute (UGR project, University of Edinburgh)

Mindi Price (UGR, Rijksuniversiteit Groningen)

Christina Protopapa (UGR project, Cardiff University)

Nelly Rahmede (UGR project, Rijksuniversiteit Groningen)

Lee Robson (UGR project, University of Edinburgh)

Gintare Siugzdinyte (UGR project, University of Edinburgh)

Teddy Spassova (UGR project, University of Edinburgh)

Dr.~Michael J. Wolff (UGR project, Rijksuniversiteit Groningen)

Dr.~Berry van den Berg (UGR project, Rijksuniversiteit Groningen)

Yixia Zheng (UGR project, Rijksuniversiteit Groningen)

\chapter{Preparing for a stellar project}\label{intro}

You may be joining the lab hoping to publish some of your work. We
definitely aim for the work we carry out to be good enough to promote to
colleagues around the world. We always want our work to be convincing
and to make an impact. This section lays out our guiding philosophies
for making that happen.

\section{Choosing a research
question}\label{choosing-a-research-question}

Researchers just beginning often struggle with choosing a good research
question. That's no surprise: this is really difficult. There are a lot
of factors to consider, both in terms of logistics and potential impact.
Under pressure to find a question, the junior researcher might settle
for a trivial extension that does not have much theoretical depth (e.g.,
replicating an interesting finding in a different population).
Extensions like that are not necessarily uninteresting, but they usually
lack theoretical drive (e.g., is there a good reason to think that the
finding would differ in that population?). The strongest research
questions also lead to one of a small set of possible conclusions, not
just did it or did it not work. Ideally, you want to be clear about what
the competing predictions are, and develop a design where you could
learn something even from finding a null effect.

It is also difficult to estimate how much time a project will take. Even
with many years' experience, I cannot do this reliably.
Enthusiastically, young researchers starting their first project often
come up with huge, impractical ideas that would be impossible to
complete under their time and resource constraints.

You should definitely think about what research questions interest you,
and you should even think grandly about what you would like to
accomplish. But after you have done that, let your PI help. Your PI will
have the background knowledge you currently lack to identify ways to
refine your ideas and give them more theoretical depth. She will be able
to suggest directions you can look into. She will know whether what you
are considering is feasible or not (but beware anyway, she is way too
optimistic).

The ability to hone in on a focused and interesting question will
develop as you accumulate knowledge. I definitely couldn't do it as an
UG, and I knew at the time I couldn't. To get a start on my honor's
research, I asked one of my mentors for help. After listening to
\href{https://psy.fsu.edu/faculty/kistnerj/kistner.dp.php}{Prof.~Janet
Kistner} give an informal presentation at a lab meeting, I approached
her about doing an UG dissertation (it was optional at FSU). My starting
pitch was simply that I found her group's project about the development
of motivation interesting, and that I would like to learn more about it,
and did she think there was any possibility to carry it further with an
UG dissertation project? I wasn't confident that I could independently
identify a good way to contribute, but I could explain why this research
appealed to me (it was the clever operationalization and manipulation)
and indicate that I wanted to learn more about it. So, you do not need
to have a brilliant idea ready - you just need to think about your
interests and goals enough to express them to your PI. Then together you
will weigh options and refine them.

Shifting to the PGR level, you have often identified your topic before
you even arrive on campus. Some brief advice for prospective PGRs
(because I get a lot of oddball requests): when you are seeking
supervision as a PGR, you should be seeking someone who can evaluate the
research you want to do. Never spam PIs from whole departments with your
request for supervision. Do not bother cognitive psychologists with your
proposal about dentistry, or agricultural business administration, or
using Western blot (I've seriously received an earnest, seemingly
personal email from a student who wanted to visit my lab to learn
Western blot techniques!). Put in the effort to identify potential
supervisors who actively do work in the field you are proposing to work
in. You are unlikely to even get a response from the others. Even if you
do, it will just be confusing (uh, what makes you think I can teach you
Western blot?).

Assuming you have chosen this lab for good reasons, the first step on
arrival at your PGR lab is to discuss your PGR proposal with your
supervisor. The proposals we request for admitting students are short
and lack much detail. They might contain good ideas, but they are not
really plans of action. First thing: sit down with your PI and flesh out
a more thorough and detailed plan.

There are three main principles that I try to work into any research
project:

\begin{enumerate}
\def\labelenumi{\arabic{enumi}.}
\item
  Try to think of a design that can offer you more than one way to
  address your point.
\item
  Think about minimizing measurement noise.
\item
  Show all your final work, preferably publicly.
\end{enumerate}

I'll elaborate on the first two points now. There is so much to say
about the third point that there is a whole chapter about it later.

\section{Planning for converging
evidence}\label{planning-for-converging-evidence}

There are many ways to write an adequate scientific article. Often,
scientific articles consist of only one experiment, analyzing one
dependent variable. Sometimes a report this simple is striking and
clean, and all you wanted to know about that topic. However, usually
such a report leaves a lot of unanswered questions. In the best-case
versions, the single-experiment paper reports a beautifully designed
study, with obvious predictions, clear results, with results coming from
multiple dependent measures and all converging on the same answer. I
don't believe that these lovely, succinct papers come from single-shot
attempts. I think they come from extensive piloting, and eventually the
judicious decision to write a brief paper about the ultimate design
rather than write a long paper including every step that led to the
ultimate design. While we will never hide the preliminary work that went
into a published paper, the ultimate communication we make will reflect
the evidence that we are confident about: experiments conducted after
piloting, when we are sure we understand how the task is working. Though
you should not assume that all the data you collect will be publishable
quality, or that you will write a paper about each experiment you
conduct, you will save yourself some effort though if you consider in
advance what kinds of evidence might help you make a converging case,
and how much of that evidence you can work into your experimental
designs.

One important tip is to always measure and record the rawest dependent
variables possible. For example, we are always measuring accuracy in our
work. With accuracy, you can record whether a given response was correct
or incorrect. That might be all you are planning to analyze, but it
limits you. It might be interesting to know in what way a wrong response
was wrong. You might not realize it at the start of your project, you
might later think of a way to use errors to determine between two
otherwise vague possibilities. By recording only ``right'' or ``wrong'',
you are losing information that might have been useful.

I've regretted decisions about this myself. I have a few data sets
measuring spatial serial order reconstruction via mouse click responses.
My program recorded a lot, but I did not manage to get it to record
response times for every mouse click. There are exquisite analyses that
can be done with this information (e.g., Chevalier, James, Wiebe,
Nelson, \& Espy, 2014) that I cannot perform on those data sets (e.g.,
Morey \& Miron, 2016; Morey, Mareva, Lelonkiewicz, \& Chevalier, 2018).

Our resources are always limited: there is never as much money, or time,
or participant energy as we could spend. Make the most of your resources
and record as much as you can about your participants' response.
Sometimes it will take extra effort in programming your experiment, but
if you get to N=40 and then decide you want those data, you will not
only have to figure it out, but you will have to begin collecting data
again. Figure it out first, or at least weigh the risk of recording less
with the PI before beginning data collection.

Sometimes, convergence of evidence will only be possible with multiple
experiments. Assuming that we have the resources, we want to write one
very convincing paper rather than several papers with evidence that
doesn't support a complete argument. This also lets you better control
your narrative; writing about one isolated and interesting result, you
will need to discuss the myriad possibilities for interpretation that
you have not yet ruled out. Putting multiple lines of evidence in one
paper means that you can rule out more interpretations, which
strengthens the interpretation your data support.

It is unlikely that you will be running an experiment, then publishing
it, and repeating that process, so you want to plan for serial data
collection, for efficiently processing results, for being ready to
collect data during convenient seasons for participants, etc. if you
want to write convincing papers that are bursting with brilliantly clear
evidence.

\section{Minimizing noise}\label{minimizing-noise}

Another factor that separates the so-so evidence from the convincing
evidence is how clear and obvious the effect looks. Of course we will be
doing inferential statistics, but ideally these only confirm what is
already obvious in your plots. This means you should collect more data
than you think you need to detect an effect of a particular size (more
on that later). You should also think carefully about all of the
irrelevant factors that could affect your participants' responses and do
what you can to neutralize them in your experiment's design and
protocol. These are things like interruptions (phones beeping in the
middle of a task, fire alarms sounding), irritating background noises,
fatigue, contingencies that unintentionally reinforce random responding,
or individual differences beyond those we normally expect (e.g.,
participants under the influence of a drug). Much of our lab's standard
procedures have been adopted through experience and practice to minimize
extraneous measurement error. Adapting your protocol from an existing
one helps you benefit from someone else's experiences.

\section{Illustrations of scope and output of student
projects}\label{illustrations-of-scope-and-output-of-student-projects}

It can be difficult to imagine what a reasonable project looks like if
you haven't done one. It is also difficult to generalize enough to say
what a great UG dissertation or PG thesis looks like. I'll try though.
Natually every project is custom, but I've attempted to illustrate a
holotype - the modal characteristics of a solid, successful project -
for UG and PG projects.

\subsection{Holotype: UG dissertation timeline and
workload}\label{holotype-ug-dissertation-timeline-and-workload}

In the UG dissertation scheme at Cardiff University there are milestones
built into the system. You must have a working title after about 2
weeks, and you must have written a short working abstract after about 4
weeks. There is little time for thinking about what research question
you are asking and how you will answer it, because by the end of October
you need to be implementing. Start off in week 1 (or earlier if it suits
everyone) by meeting with your PI and informally chatting about your
interests and what skills you hope to learn during the project. Your PI
will have a mental short list of possibilities and can quickly see which
ones are most likely to suit you, and how they might be adapted to suit
you better. You should leave this meeting with a few ideas and a list of
readings.

Your project will be one experiment with at least 30 participants. If
you are working collaboratively with other students, you might be
running a series together or a larger, more complex design that requires
a larger sample.

Once you have settled on a question, find your deadline, work backwards,
estimate how much time you want in between major milestones (e.g.,
getting feedback from your presentation and submitting, writing after
finishing your analyses, analyzing your data after collecting them,
etc.). You will probably find that you want to have at least begun data
collection before winter break. This is also a bit early in terms of
when most UG students collect data, so you will have plenty of
participants.

This gives you one month to implement your experimental design and
develop your protocol. Here is where your PI can help. There will be
lots of ready-programmed experiments available from other lab projects.
Your PI can help you limit your choice of paradigm to one that the lab
already uses in some form, identify modifications you may need to make,
and help you see how these modifications fit with what has been done
previously. You will learn a little about programming, but unless you
already know how, it will be extremely hard to learn to program and
create a useful program from scratch in a month. Before collecting data,
you will have already:

\begin{enumerate}
\def\labelenumi{\arabic{enumi}.}
\tightlist
\item
  Written an experimental protocol explaining step-by-step \emph{how} to
  run a session
\item
  Tested at least one pilot participant to make sure your protocol
  includes every step needed
\item
  Gone over the output from your program with one of the senior
  researchers to make sure it records everything you need to answer your
  research questions
\item
  Scheduled a real participant for a time when one of the senior
  researchers in the lab can supervise and check that your
  implementation of the protocol is fine
\item
  Created an Open Science Framework page for your experiment (with the
  PI as a full-fledged user), uploaded your experimental software, your
  protocol, paperwork, and a written description of your method and
  planned analyses
\end{enumerate}

When you return from winter break, you will finish your data collection
and begin processing your data. Here is another moment to seek advice
from the PI. Our lab manual codifies many useful procedures for
processing data, but they require understanding R scripts well enough to
copy and modify existing scripts. Your supervisor can help you with
this, and direct you to ways to learn how to get specific things you
need from the data. You should complete our standard anonymizing steps
on your compiled raw data (either on your own or with help from a senior
lab member), and then save the resulting data file to your Open Science
Framework page. You may now analyze the data and prepare for your March
lab meeting presentation, where you will receive feedback on your
analyses. After that, you have a few weeks to finish writing.

\subsection{Holotype: PhD thesis timeline and
workload}\label{holotype-phd-thesis-timeline-and-workload}

The PhD thesis is a substantial body of work carried out over three
years. A good series of PhD studies should produce at least one
substantial journal article. Usually, a single ``chapter'' of the PhD
includes the contents of one published or publishable paper. Such a
chapter may contain a single experiment or a series of related
experiments. A good PhD thesis will contain at least three empirical
chapters, plus an introduction chapter and a concluding chapter.

A PhD thesis must be cohesive. It cannot include chapters about
different things, because the introduction and concluding chapters must
make clear how the work fits together. So, you must stick closely to
whatever topic you and your PI agree on when you begin.

It is a good idea to identify tasks that you can complete in parallel so
that you are not reactively collecting data, seeing how it went,
deciding what to do, and repeating the process. I think it is ideal to
jump-start the PhD project by simultaneously working on a related lab
project while designing your own first project. This gives you hands-on
experience with the lab procedures you will be using so that you can
start your own project quickly when it is ready, and potentially
involves you in one lab publication besides any that come from your PhD
experiments. An ideal timeline could involve starting with related data
collection upon arrival, discussing a way to meta-analyze your topic
with your PI and working on that alongside data collection, and then
using the knowledge acquired from meta-analysis to plan your first
experiments. You want to be ready to collect some data in the spring,
but to plan for the bulk of your data collection to occur in the
academic Year 2 and the fall of Year 3, when participants are widely
available. Ideally, January - September of Year 3 is for writing,
knowing that you could probably collect another experiment in the spring
semester if you think it would help.

Each empirical chapter should be a self-contained empirical paper (that
perhaps has been submitted for publication). You will have been
preparing these gradually during Year 2 and the beginning of Year 3, so
some of the months reserved for finishing writing will be spent
implementing feedback from peer review. Ideally you will also have some
time to consider preparing applications for post-doctoral research
posts.

\section{Making the most of your
time}\label{making-the-most-of-your-time}

Do those timelines sound reasonable? Experience has taught me that there
is somehow never enough time. Every hour you can save by relying on
existing procedures and pipelines will help you make the most of your
effort. You will also naturally find ways to innovate our pipeline as
you work, so as others have paid it forward to you by programming
experiments you can modify and figuring out useful R functions and
scripting processes that you may copy, you will pay it forward by
periodically incorporating something you have discovered into our
process.

\chapter{Lab meetings}\label{lab-meetings}

Every two weeks (unless there's something unusual going on), our whole
group shall meet to discuss our ongoing work, progress, plans,
logistics, etc. The content of each meeting will be arranged and
communicated ahead of time. All lab members are expected to attend lab
meeting if at all possible. Lab members should avoid scheduling anything
during lab meeting time.

\section{Time and place}\label{time-and-place}

Every other Monday (beginning 15 October 2018), 16:10-17:30, 7.10 Tower.
Lab meetings will definitely occur during term time. They may occur
during some term breaks depending on who all is working across the
break.

\section{Standing agenda}\label{standing-agenda}

\begin{enumerate}
\def\labelenumi{\arabic{enumi}.}
\tightlist
\item
  Logistics: Is anything broken? Are there scheduling
  conflicts/difficulties? Programs to troubleshoot?
\item
  Progress report: PI will ask a selection of members for a progress
  report
\item
  OSF Review: We'll check whether any milestones have been reached
  lately
\item
  Research content (ordered by priority; we won't do all things every
  meeting): Lab members starting a new study will give their elevator
  speech and demo their program, lab members with new data will describe
  it for us (must have a graph!), we will discuss a relevant research
  paper
\end{enumerate}

\section{What's in it for everyone}\label{whats-in-it-for-everyone}

You may wonder why we do this. One reason is that it is an efficient way
for the PI to get information from everyone. The PI will see lab members
on other days and times too, but it is useful to know that there is a
specific, repeated occasion when we will regularly see each other.

Another reason is that it is unhealthy for researchers to struggle alone
with difficulties. We prioritize trouble-shooting in lab meetings so
that anyone experiencing a problem has a regular opportunity to talk
about it with a supportive group of people. It is also useful for others
to hear about these problems even if they are not directly affected
because 1) it normalizes the fact that everyone struggles with
logisitics and programming sometimes, 2) the same problem may impact you
eventually, and by attending to others' concerns in lab meeting, you
will at least be familiar with it.

We also need a regular moment to mark the interim successes that occur
during research projects that don't merit wide attention. Finishing a
pilot-testing, submitting a conference presentation, finding something
interesting in an analysis are all moments that call for some
celebration, but will only be interesting to like-minded geeks.

Finally, you may not yet personally see how another lab member's work
relates to your's, but it does. Hearing about this work and reading
papers together builds our common knowledge space and affords
opportunities for generating new ideas. I hope these meetings will
always be interesting. Inevitably they will sometimes be boring. But
please make the effort to attend and participate. Though you may
sometimes not see the immediate gain, you never know when our meetings
bring some opportunity for you to improve your project or contribute to
someone else's.

\section{Project-specific meetings}\label{project-specific-meetings}

Sub-groups of us shall sometimes meet for specific business that doesn't
involve the whole lab. For instance, many lab members are students, and
students may have contact hours with the PI related to their course that
non-student lab members won't attend. The personnel involved in
particular projects will convene \emph{Data Round-up Meetings}, in which
we compile, anonymize, exclude ineligible participants from a data set,
and securely back-up the data ahead of the analysis.

\chapter{Seminars at Cardiff
University}\label{seminars-at-cardiff-university}

One of the perks of working at a university is that you are expected to
spend some of your time developing your interests and participating in
university life. You are encouraged to attend seminars whenever
something that interests you is on offer. You may want to keep the times
of these regularly-occurring seminars free during term time.

\begin{itemize}
\tightlist
\item
  School of Psychology Colloquium: Fridays from 14:10-15:00, usually in
  12.11 Tower
\item
  Developmental / Environmental / Social / Cognitive Seminar: some
  Wednesdays, 12:10-13:00, 12.11 Tower
\item
  Open Science Working Group: Join the mailing list (maintained by Chris
  Chambers,
  \href{mailto:ChambersC1@cardiff.ac.uk}{\nolinkurl{ChambersC1@cardiff.ac.uk}})
  and watch for announcements. Meetings usually include a catered lunch
  and require an RSVP.
\item
  Cognitive Development Group: some Thursdays, 14:00-15:00, 3.11
\item
  Memory Journal Club: monthly, watch for emails, time and location
  vary. One person presents a paper for discussion. Content is usually
  neuroscience.
\item
  Brain Mapping Seminar: Mondays, 12:10-13:00, CUBRIC Seminar Room 2
\item
  Postgraduate Internal Seminar Series: Thursdays, 13:10-14:00, location
  alternates between lounge on 5th floor of Tower and CUBRIC Seminar
  Room 2
\end{itemize}

\chapter{Student projects we support at Cardiff
University}\label{student-projects-we-support-at-cardiff-university}

\section{Postgraduate researchers}\label{postgraduate-researchers}

\subsection{PhD projects}\label{phd-projects}

We currently consider applications for PhD projects. One line, to work
on a PhD project based on the PI's main research interests, is currently
available. The successful candidate would begin in October 2019. Two
ideas for projects are on offer, but it is likely that only one
candidate, working on one of these projects, may be funded. This
position is only available to students with access to local fees (e.g.,
UK/EU residents).

\textbf{Explaining interference in working memory}: Morey has
demonstrated that visual information is fragile, and is disrupted by
many kinds of cognitive tasks, not only visual or spatial tasks (see
Morey (2018) for a summary of the evidence). This contrasts with verbal
information: there is a lot of robust evidence that verbal information
may be maintained nearly as well when doing an irrelevant non-verbal
task as when focusing only on the verbal memory task. The successful
candidate will work with the PI to design a series of experiments
comparing alternative explanations for this pattern, and delineating the
boundary conditions of this pattern to better inform theory. The
candidate will have access to preliminary pilot data to jump-start his
or her work.

\textbf{Maintenance of spatial information, with and without intent}:
Memory for novel, arbitrary collections of spatial locations appears to
be much more limited than memory for verbal information. Yet some
evidence suggests that spatial information of verbal labels or visual
content is autmatically encoded and maintained at least for a short
period. The successful candidate will apply new electrophysiological
methods, using scalp alpha readings to infer which spatial locations are
maintained, to compare spatial location memory under a variety of
circumstances. The successful candidate will work with the PI to design
a series of experiments comparing alternatives. The candidate will have
access to preliminary pilot data to jump-start his or her work.

This opening is funded by the School of Psychology. Details on how to
apply will be available in late 2018 or early 2019. The School of
Psychology expects candidates to have minimally a 2:1 Bachelor's degree.
Realistically, short-listed candidates for School-funded PhDs tend to
have a first-class degree (either Bachelor's or Master's, or both). The
basic position is a 3-year funded PhD, but exceptional candidates will
be considered for a Graduate Teaching (GTA) position. GTAs perform light
UG instruction alongside their PhD work and receive increased funding
for 4 years.

There is also an opportunity to undertake a PhD in meta-scientific
research, supervised jointly by Dr.~Candice Morey and Prof.~Chris
Chambers. See our
\href{https://www.findaphd.com/search/ProjectDetails.aspx?PJID=100639}{advert}
for details.

There are other routes to funding a PhD that prospective students may
also explore. Cardiff University offers limited funding for
international student fee waivers and stipends. Some opportunities are
general, apply to students from anywhere in the world, and there are
some scholarships available to prospective students from particular
countries.

Several UK funding agencies (including Wellcome, ESRC, and Leverhulme)
offer PhD studentships. These opportunities tend to be student-led: the
student is expected to write the proposal and application. Even so, it
is a big advantage to have the advice and assistance of the potential
supervisor in crafting an application for a PhD scholarship. Dr.~Morey
will assist excellent prospective students with applications provided
that 1) the research proposed aligns closely with her research
interests, 2) the student demonstrates capability to carry out a PhD
project (e.g., has a strong prior academic record, strong communication
skills, and excellent references from previous mentors), and 3) the
prospective student makes contact in enough time to develop and prepare
the proposal ahead of the funder's deadline. (Note: ``in time'' for
something like this is, realistically, at least a full month before
anything needs to be submitted.)

\subsection{DEdPsy, Educational
Psychology}\label{dedpsy-educational-psychology}

Some of the developmental work we do could serve to support the research
training of DEdPsy students. Please get in touch with Dr.~Candice Morey
if you are interested in hearing about project opportunities and
discussing the educational relevance of our basic memory research.

\subsection{Taught MSc student
projects}\label{taught-msc-student-projects}

We are happy to host students taking part in the \textbf{Neuroimaging
Methods and Applications}, \textbf{Children's Psychological Disorders},
and \textbf{Social Science Research} MSc degrees. Appropriate project
suggestions will be available annually. Feel free to get in touch with
Candice Morey to discuss ideas.

\section{Undergraduate researchers}\label{undergraduate-researchers}

\subsection{Final year UG projects}\label{final-year-ug-projects}

Every year we host final year UG projects. A selection of project
descriptions are available in the course handbook every spring. We often
offer both developmental and cognitive opportunities. Upon joining the
lab, students receive an updated list of opportunities for projects.
Note though that some projects require the participation of mulitple
student researchers. Choice of project is always a function of both
logistic contraints and student preference.

\subsection{Intercalcated Psychology and Medicine student
projects}\label{intercalcated-psychology-and-medicine-student-projects}

Some of our basic research is suitable training for Medicine students
spending a year learning Psychology. Project options for intercalcated
students are sometimes made available in the course handbook.

\subsection{Professional placements}\label{professional-placements}

In some years, we may have the capacity to offer a professional
placement to students on a Psychology with Professional Placement
course. These positions will be advertised at eligible institutions.
Interested students may enquire if they do not see an advertisement in
their programme; we may possibly be able to accomodate a placement
student even if we have not advertized.

Our placement will be entirely research-focused. This may not fulfill
the requirements of all placement programmes.

\subsection{CUROP and SPRINT summer research
projects}\label{curop-and-sprint-summer-research-projects}

Cardiff University and the School of Psychology offer funded summer
internships for current students. Interested students should get in
touch with Dr.~Morey as early as possible in the spring. Project
descriptions will be available in the spring.

\section{Pre-university students}\label{pre-university-students}

\subsection{Nuffield summer intern
placements}\label{nuffield-summer-intern-placements}

We are willing to offer research experiences to teenagers interested in
learning about cognitive and developmental psychology some summers. When
we have an opening, we will list it in the relevant places.

\chapter{Lab spaces we use}\label{lab-spaces-we-use}

We do basic cognitive research with young adults and children as
participants. Sometimes we make use of special equipment (e.g., eye
trackers, EEG). These needs mean we have a stake in several laboratory
spaces within the School of Psychology.

\section{64 Park Place, room 0.05}\label{park-place-room-0.05}

This lab is available for use by anyone in the School of Psychology, but
in practice it is used mainly by us. The room contains two
sound-attenuated chambers with work stations both inside and outside the
chambers. Priority for using this space: studies with adult participants
in which participants must run singly, and for which noise reduction is
very important. (This describes most of our research.)

\subsection{Access to 64 PP 0.05}\label{access-to-64-pp-0.05}

Request swipe-card access from the PI, who will forward your request to
\href{mailto:psychlocks@cardiff.ac.uk}{\nolinkurl{psychlocks@cardiff.ac.uk}}.

\subsection{Security in 64 PP 0.05}\label{security-in-64-pp-0.05}

64 PP is only accessible to personnel with swipe-card access. The
Cardiff University Film Unit is based there, several staff have their
offices there, and many PhD student's desks are in the building. During
regular working hours, there should always be people around to help in
case of an emergency. Note the signs indicating fire escape routes and
current first aiders.

As with any building that has restricted access, you should not allow
anyone to follow you in when you enter, and you should not admit someone
requesting to enter if you do not know who they are and what they are
there for.

\subsection{Booking 64 PP 0.05}\label{booking-64-pp-0.05}

Whether you are programming your study, pilot-testing your study, or
running participants, you need to schedule any work time in 64 PP 0.05
or risk being kicked out because another person scheduled it properly.

We maintain an Outlook calendar for this lab so that you may
conveniently book it. You may request access to the calendar from the
PI. Once granted access, you will be able to view and edit the calendar
``64 Park Place 0.05'' from your university MS Outlook account.

In general, you may book the lab any time it is apparently free. But do
keep in mind that other people may need to use it too, especially during
peak participant-running times.

We want to always faciliate efficient use of the space. Please consider
the following when booking:

\begin{itemize}
\item
  You should make separate reservations per booth so that we can tell if
  both or only one booth is in use at a particular time.
\item
  Don't ``block-book'' the booths to run particpants for a long period
  (i.e., a whole day). You may actually use the room for a long period
  like a whole day, but make a separate reservation in the calendar for
  each discrete event (e.g., a 1.5-hour session with a participant).
  This way, if some of your bookings are not taken or cancelled, you can
  update the calendar so that the rest of us can see if there are times
  when the booths might be free.
\item
  Besides the date and time, your reservation should include your name,
  and information about what you are doing (e.g., ``participant'',
  ``programming'', etc.), and which booth you are reserving. This
  information helps other users know whether they may be able to step in
  during the booking, and in the event of a mis-communication about
  scheduling, helps prioritize.
\item
  If booths somehow become over-committed, priority will go to running
  scheduled participants over programming or preparing. During peak
  times, plan to do any programming or preparing somewhere else, and
  when you need to use the lab to install your programs and test them,
  plan to do that during a time that is inconvenient for participant
  testing.
\end{itemize}

\subsection{Peak usage}\label{peak-usage}

The lab will be busiest during term times. That's mid October until mid
December, mid-January until Easter break, during the regular work day
(e.g., 9:30 - 16:30). Whenever possible, do not plan on having access to
the lab for programming during peak times. During peak times, the best
times to use the lab for preparing a protocol, installing programs, or
retrieving data are early in the morning (e.g., before 9:30) and during
times when both Year 1s and Year 2s have lectures. For Autumn 2018,
that's Thursdays and Fridays from 10:00-11:00.

\subsection{Storage}\label{storage}

There are shelves in the lab which you may use to store things that you
use while in the lab (e.g., experimental paperwork, notebooks, etc.).
Bring some sort of container to keep your things in, and label it with
your name. After you complete a study, you should remove the paperwork
associated with that study from the lab. Please do not leave personal
items in the lab after you are finished working there. When you are no
longer collecting data for the lab or working in 0.05 regularly, remove
the things you have stored there. The contents of the shelves will be
periodically binned to avoid clutter, and anything associated with a
former lab member that's left behind will be binned or appropriated by
someone else.

\section{CUCHDS}\label{cuchds}

\url{http://psych.cf.ac.uk/cuchds/}

\section{Laboratories in CUBRIC}\label{laboratories-in-cubric}

\subsection{EEG}\label{eeg}

\url{http://www.cardiff.ac.uk/cardiff-university-brain-research-imaging-centre/facilities/electroencephalography-labs}

\subsection{EyeLink}\label{eyelink}

\url{http://www.cardiff.ac.uk/cardiff-university-brain-research-imaging-centre/facilities/cognitive-testing-labs}

\chapter{Working transparently on
OSF}\label{working-transparently-on-osf}

In Chapter 1, I described how we are working together on something
bigger than any single project. My role is both to facilitate your
project while you are completing it and to carry it forward when
possible after you are finished with it. To do both of those I things, I
need to be able to see your work. I need to know what you have finished
(and by extension, what you are currently working on) and I need to be
able to find your finish products and comprehend them even if I look
them up years after you finished them. That means that you need to
curate your work in some place that I can always access. By
\emph{curate}, I mean create support that guides me through your work so
that I, or any one else, can figure out exactly what you did.

It is a chore to curate the materials and data from a project after it
is already finished. You forget quite a lot even in the months between
starting and finishing a short project. However, while you are working
on the project you are actually doing the work I mean when I say
\emph{curating}: you frequently need to explain to me or other
colleagues how your experiment works or what you did to the data. So,
what I insist on is that you do this in a permanent way by putting this
information on \textbf{\emph{OSF}} (that's for Open Science Framework),
rather than in a private way like jotting it down on your private
notebook.

\section{The OSF environment}\label{the-osf-environment}

The OSF is a web-based repository that scientists use to organize their
research. OSF is free to use. Sign up for an account on OSF by visiting
osf.io and clicking the green \textbf{Sign Up} button at the top of the
page.

Depending on what you are working on, you may need to create a new
project or you may be added as a Contributor to an existing project.

\subsection{Creating a new project}\label{creating-a-new-project}

Most projects we do will share a common structure: they will include
directories for preliminary planning, experimenter paperwork, run files,
data and analysis scripts, and communications. If your new project will
contain most of these, then you can start by searching OSF for the ``OSF
Research Project Template'' designed by Candice Morey. You can look at
the descriptions of the components, and if you like, duplicate this
template and use it for your new project. To duplicate it, click the
button on the upper right side of the screen that looks like a little
branch. A menu will appear below.

\includegraphics{osf_fork.png}

If you choose ``Duplicate this template'', then OSF will create a new
project with the same directory structure. You can then fill in the
details, delete directories you may not need, or change the names of
directories. If you choose ``Fork this Project'', OSF will create a new
project copying everything about the forked page, including the Wiki
content and the project description. All of that can be edited in your
new project.

Remember that your project, though new to you, may be embedded with some
related ongoing project. If so, it might make more sense for you to be
added to that existing project as a contributor. Identifying which
circumstance you are in will be one of the first things you and the PI
sort out.

\subsection{Becoming a contributor}\label{becoming-a-contributor}

One of the lab members may add you as a contributor to an existing OSF
page. This will enable you to at least read, and probably read and edit,
the existing page. Our projects usually have the same directory
structure, but the number of sub-directories in each section may vary
depending on how many studies fall under one project heading. If you are
added to an existing project, first familiarize yourself with what is
there and how it is organized. Then, work discuss with one of the other
contributors how you should add to the page. Most likely, there will
sub-directories in each folder corresponding to each related project,
and you will be asked to create sub-directories for your project.

\subsection{Naming conventions}\label{naming-conventions}

For the files you see on OSF, there should be some commonalities in how
they are named. You will need to come up with a scheme for naming the
files associated with your project. This should not be something too
generic like ``dissertation consent form.doc'' or ``experiment 1'': if
that were allowed we would have dozens of files with the same names.
Pick a short name or acronym for your project that says something about
what is unique about it. Then, use that short name to name your files.
So, instead of ``consent form.pdf'' your file might be
``wmSpanPrefix\_consentForm.pdf''. Note also the use of the underscore:
it is good practice never to include spaces in file names. Note also the
initial word is lowercase and subsequent words are capitalized. This
formatting is called \emph{camelCase}. We frequently have to type file
names in contexts where the name must be reproduced exactly for code to
run, so it is very helpful if you consistently apply camelCase to all
your final names.

\subsection{Contributors}\label{contributors}

You can see who all the \emph{contributors} on a project are by clicking
the \textbf{\emph{Contributors}} button on the upper ribbon. You may be
able to add new contributors if you would like for someone else to be
able to see your page. You can choose whether to let a new contributor
only read your page or also edit your project page.

At minimum, you and your PI should both be contributors to your project.
The PI should always have at least Read + Write access, and possibly
Adminstrator access.

A \emph{bibliographic} contributor is someone who will share in
authorship on any presentations or papers that result from the project.
If you want to share your page with an outside colleague (maybe as part
of a job application, or maybe with a peer who is reviewing your work
for you) you can add them as a non-bibliographic contributor. If this
person only needs to have a look at your project, then alternatively you
can provide them with a view-only link without adding them to the
project as a contributor.

\subsection{Public versus private}\label{public-versus-private}

Usually, your project will start out as private, and eventually may
become public. Our standard is to set a project to ``Public'' once we
have communicated the data outside the lab by presenting it at a
conference or submitting a manuscript.

There may occasionally be good reason never to make a project public,
for instance if we cannot ensure that our participants' identities are
anonymous. However, normally we can easily do this.

\subsection{Wikis}\label{wikis}

OSF allows you to describe each project directory and sub-directory with
a Wiki. Take a look at the Wikis on the OSF Research Project Template
for an example of what you might typically want to know about the
contents of a directory.

\section{Using your OSF project page to monitor
progress}\label{using-your-osf-project-page-to-monitor-progress}

We will both be able to clearly see how your work is progressing from
the updates on your OSF page. The pace of progress will vary but the
order of events will be pretty consistent. I'll briefly describe the
usual project components in the usual order. Each of these steps has its
own chapter with more detail later.

\subsection{First you generate a plan}\label{first-you-generate-a-plan}

When we first start discussing the project, we will converge on an
experimental design and will work out the possible outcomes we might
observe given that design. This may take a while, through a process of
reading and discussion. Eventually, you should aim to note down your
plans and predictions in a document, and post this document in your
Preliminary Plans sub-directory.

Your Preliminary Plans directory may also include work you did that
informed your plan. For example, see the Preliminary Plans directory in
\emph{A complex effect of memory load on processing speed}.

\subsection{Then you prepare your
materials}\label{then-you-prepare-your-materials}

You will need to get, modify, or create software to display stimuli and
collect responses, craft a consent form and debriefing statement for
your participants, work up an experimental protocol, and test that these
materials work to all the contributors' satisfaction.

Once we are happy the materials (i.e., they have been pilot-tested, the
output has been checked, etc.) they go in their appropriate
sub-directories.

\subsection{Then you work on data
analysis}\label{then-you-work-on-data-analysis}

Once you have stopped acquiring data, you should compile it into one
file (or we may do this together in a \emph{Data Round-up Meeting}),
anonymize it (more on that later), and upload it to the Data and
Analysis Scripts directory. You can add analyses when you have finished
them.

\subsection{Finally, you communicate your
research}\label{finally-you-communicate-your-research}

When your written report or presentation is complete, you upload it to
your Papers and Presentations directory.

At any time during your project, I will be able to see what is on OSF,
and with that know what you are finished with, what you should be
currently working on. Knowing this will help me give you useful feedback
at the appropriate time and be better prepared to help with whatever you
might be grappling with.

\section{Registering milestones}\label{registering-milestones}

OSF provides free storage for your materials and data. You can think of
it as a back-up system that encompasses not just your data, but also
your thoughts about your project. That turns out to be really important
because we need to keep track of what we believed before seeing the data
versus what we believed based on the data.

Making a \emph{registration} creates a frozen, time-stamped version of
your project. Registrations cannot be changed, while projects can.
Someone could remove data from a project, but not from a registration.
It makes sense to register your project at key moments where you might
want a permanent record of what you thought before the next phase.
Generally, we want to register a project:

\begin{enumerate}
\def\labelenumi{\arabic{enumi}.}
\tightlist
\item
  When we have recoreded the definitive study design and predictions
\item
  After the anonymized data are uploaded
\item
  When the analyses are finished
\item
  When the manuscript is submitted (repeat as necessary)
\item
  When the manuscript is published
\end{enumerate}

These registrations back-up your work, but also allow you to prove when
your hypotheses preceded the data.

\section{When copying is okay}\label{when-copying-is-okay}

You may be added to an ongoing project, and assume responsibility for
some new component of that project. The project might already include
materials you could use. But is it a problem to just use them, rather
than create your own from scratch?

There are some research materials that it is always okay to copy, or
modify slightly for a new purpose. You should always make your consent
form from the lab's template. Doing this ensures that important legalese
that should be in the form is always there. You do not need to come up
with new, original ways to program the same experiment or script the
same analysis; it is always fine to copy an existing sample and modify
it slightly to meet your needs. You should acknowledge the original
creator when doing this, if the original creator is not already a
bibliographic contributor on your project (if s/he is, then s/he will
already get acknowledged in your work).

You should not absolutely copy text destined for your manuscript, if you
are producing a dissertation or thesis based on the work. Dissertations
and theses differ from group-written manuscripts in that a student is
composing them to show what s/he has learned about the work and about
writing an empirical report in general. If your project is a
dissertation or thesis, you must write your work originally, not quote
or refer heavily to the method or result from a related lab project. You
may look at any example documents we have to see what the other authors
included, but then close the example and write your own.

If you are working on a project that follows on from another one and is
not being submitted for a university degree, then you probably could
copy text from related Method sections and modify it as needed. In a
serial report of several experiments, the Method and Results of
experiments that follow from previous ones are often streamlined by
referring back to the first. This is fine for a manuscript intended for
publication; in fact, you want to aim for brevity when possible.

\chapter{Research ethics}\label{research-ethics}

\section{Basic principles}\label{basic-principles}

Scientists have done appalling things for the sake of advancing
knowledge, such as intentionally infecting people with diseases, robbing
graveyards for corpses to dissect, and leading participants to believe
that they had critically injured someone else. To prevent scientists
from committing such violations (as well as less dramatically awful
ones), international conventions now protect the health, privacy, and
autonomy of people participating in scientific research.

Researchers should be familiar with the Declaration of Helsinki.
Generally, issues for concern may involve administering medical
treatments that might end up causing harm, deceiving the participants,
working with participants who are vulnerable and cannot consent on their
own behalf, and protecting participants' sensitive data from actors who
may use it for nefarious purposes. We do not carry out any research that
involves experimental medical treatment or relies on adminstration of
drugs. We also do not carry out any research that involves deceiving or
fooling the participants. We sometimes work with children (who cannot
provide consent) and though the data we collect is not obviously
``sensitive'' or useful to criminals, we acknowledge that we may not be
capable of imagining such a use for it, and are therefore under
obligation to protect participants' identities.

The research we carry out does not even approach the inhumane violations
that inspired the international conventions on research ethics.
Nonetheless there are some principles we must always bear in mind to
ensure that our participants are never made to work against their will,
never put at risk beyond the risks they experience going about their
daily lives, and always receive whatever compensation they were
expecting when they agreed to take part in the study.

\subsection{Informed consent}\label{informed-consent}

All adult participants in our studies must explicitly consent to take
part in the research. To do this in any meaningful way, they must always
be given enough information about the study to understand what
participating will be like and an honest assessment of the risks and
rewards expected from participating. All participants must be offered
this information and must indicate with a signature that they accept the
main points before they are exposed to our studies.

A template consent form is available in the shared lab OneDrive folder,
in directory \emph{templateForms}. The main things that must always be
included in the information form:

\begin{enumerate}
\def\labelenumi{\arabic{enumi}.}
\item
  A description of what the participant must do: This is from the
  participant's perspective. It does not need to say anything about what
  we are manipulating or why. It needs to describe what the session will
  be like for the participant. Usually, a session will involve seeing
  some things on a computer screen, trying to remember them, and
  indicating somehow via computer whether they remember those things or
  not. The description should be procedural and should not bias the
  participant in favor of a particular outcome.
\item
  A statement about any risks and rewards involved: Usually, our studies
  do not put the participant at any risk beyond what they experience in
  daily life. They might feel bored or tired during the session. They
  are agreeing to participate voluntarily, and depending on the
  circumstances, may receive experimental credits toward a course or a
  small amount of money for completing the session.
\item
  A statement about what we do with their data: We are committed to
  transparently sharing all of the data that support our claims with the
  public. But that does not mean that an individual participant's
  identity should be publicly exposed. We must tell participants that we
  will eventually make the data set they are contributing towards
  available, but before we do, details that could identify them
  personally will be removed. The most obvious personal details (e.g.,
  name, student id, address, etc.) we do not usually collect at all, and
  if we do, we never store them with the experimental data.
\item
  Finally, participants must be given information about who to contact
  if they are uncomfortable with any part of the research project. This
  will be the PI responsible (Candice Morey) and the administrator of
  the School Research Ethics Committee
  (\href{mailto:psychethics@cardiff.ac.uk}{\nolinkurl{psychethics@cardiff.ac.uk}}).
\end{enumerate}

\subsection{Anonymity}\label{anonymity}

Be sure to read the section in this manual about Data Security for a
full description of university and lab policy on protecting
participants' privacy. The short version is that we do not save
``personal'' or ``identifiable'' data, usually at all, and never with
the experimental data. It is our responsibility to ensure that we do not
save details that could render our data identifiable unless it is
absolutely essential, and when we do, that we use a procedure that makes
it easy to sever any link between the experimental ids and real-life
ids. We apply this procedure as soon as possible, and once it is
applied, we can no longer tell which data came from which person. Signed
consent forms must be saved for 7 years. The consent forms we save
should therefore \emph{never} include the arbitrary id assigned to the
participant's data. See also our procedures for
\protect\hyperlink{identifiers-and-data}{identifiers}.

\subsection{Educational debriefing}\label{educational-debriefing}

Whether the participants are part of the university community or from
the local community at large, part of the university's mission is to
educate. At the end of their session, all participants must be offered
information about the research presented in jargon-free language.
Participants should also be given information about where they may learn
more, and should be given the opportunity to question the researcher
working with them or the PI if desired.

\section{Applying these principles in our
lab}\label{applying-these-principles-in-our-lab}

Individual researchers will not typically need to acquire ethics
approval independently. Most research projects will fall under one of
our existing, approved generic
\protect\hyperlink{ethics-approvals-currently-in-place}{headings}.
Individual researchers will need to ensure that their experimental
paperwork, data recording procedures, and protocol are all aligned with
these principles:

\begin{enumerate}
\def\labelenumi{\arabic{enumi}.}
\item
  Written consent is always obtained.
\item
  Consent includes explicit acknowledgement that anonymized data may be
  shared online.
\item
  Arbitrary \protect\hyperlink{identifiers-and-data}{participant ids}
  are never associated with participant's signed consent form.
\item
  Any document that links personal ids with arbitrary ids must is
  treated as \textbf{confidential}. It cannot be accessible (in digital
  or print form) to anyone unnamed on the ethics application. It must be
  possible to destroy once the links are no longer needed. Once that
  destruction occurs, it must be truly impossible to re-forge the links.
  There cannot be a back-up way to reconstruct these links from any
  recorded data we preserve.
\item
  An appropriate education debriefing is prepared for every project.
  Sample debriefings are available in the shared lab OneDrive folder, in
  the \emph{templateForms} directory.
\end{enumerate}

\section{Assigning participant
identifiers}\label{assigning-participant-identifiers}

Assigning an arbitrary id number to each participant is one of the most
important bits of lab procedure to get right to maintain data security.
This is important both for ensuring that you know which data come from a
single unique participant, and for ensuring that all the data you spend
time collecting are preserved and usable.

\subsection{Ids are always unique!}\label{ids-are-always-unique}

Never re-use a participant number in the same project. Even if you know
data will be invalid (e.g., test participant, doesn't meet inclusion
criteria). This can cause confusion later, or can over-write files that
you may not want to be overwritten.

Always use the same participant number for the same person (e.g., if you
are collecting data at multiple time points from the same person).

\emph{Vignette:} An ambitious MSc project involved collecting data from
participants in two sessions. Participants ran two sessions each,
performing two of the combinations in each session. Eventually though,
all of these data needed to be analyzed together. (After all, that is
the advantage of going to the effort to get the same data in each
participant - you get to do a within-participant analysis.) The student
ran each session with a unique identifier. This certainly prevented any
data from being accidentally overwritten, but it meant that when it was
time to analyze the data, we needed a way to code which ids were
actually the same people. There is no elegant way to do this, because
there was no regularity between the ids to exploit (coding requires
regular expressions). Below is the code we used to assign new
participant ids to both of the data sets from each participant.

\begin{Shaded}
\begin{Highlighting}[]
\CommentTok{# Also, need to link subject numbers. Student ran same participants on unique id numbers in different sessions. Table of links}
\CommentTok{# 1=29}
\NormalTok{d}\OperatorTok{$}\NormalTok{Subject[d}\OperatorTok{$}\NormalTok{Subject}\OperatorTok{==}\DecValTok{1} \OperatorTok{|}\StringTok{ }\NormalTok{d}\OperatorTok{$}\NormalTok{Subject}\OperatorTok{==}\DecValTok{29}\NormalTok{]=}\DecValTok{129}
\CommentTok{# 2=19}
\NormalTok{d}\OperatorTok{$}\NormalTok{Subject[d}\OperatorTok{$}\NormalTok{Subject}\OperatorTok{==}\DecValTok{2} \OperatorTok{|}\StringTok{ }\NormalTok{d}\OperatorTok{$}\NormalTok{Subject}\OperatorTok{==}\DecValTok{19}\NormalTok{]=}\DecValTok{219}
\CommentTok{# 3=16}
\NormalTok{d}\OperatorTok{$}\NormalTok{Subject[d}\OperatorTok{$}\NormalTok{Subject}\OperatorTok{==}\DecValTok{3} \OperatorTok{|}\StringTok{ }\NormalTok{d}\OperatorTok{$}\NormalTok{Subject}\OperatorTok{==}\DecValTok{16}\NormalTok{]=}\DecValTok{316}
\CommentTok{# 4=25}
\NormalTok{d}\OperatorTok{$}\NormalTok{Subject[d}\OperatorTok{$}\NormalTok{Subject}\OperatorTok{==}\DecValTok{4} \OperatorTok{|}\StringTok{ }\NormalTok{d}\OperatorTok{$}\NormalTok{Subject}\OperatorTok{==}\DecValTok{25}\NormalTok{]=}\DecValTok{425}
\CommentTok{# 5=32}
\NormalTok{d}\OperatorTok{$}\NormalTok{Subject[d}\OperatorTok{$}\NormalTok{Subject}\OperatorTok{==}\DecValTok{5} \OperatorTok{|}\StringTok{ }\NormalTok{d}\OperatorTok{$}\NormalTok{Subject}\OperatorTok{==}\DecValTok{32}\NormalTok{]=}\DecValTok{532}
\CommentTok{# 6=13}
\NormalTok{d}\OperatorTok{$}\NormalTok{Subject[d}\OperatorTok{$}\NormalTok{Subject}\OperatorTok{==}\DecValTok{6} \OperatorTok{|}\StringTok{ }\NormalTok{d}\OperatorTok{$}\NormalTok{Subject}\OperatorTok{==}\DecValTok{13}\NormalTok{]=}\DecValTok{613}
\CommentTok{# 7=14}
\NormalTok{d}\OperatorTok{$}\NormalTok{Subject[d}\OperatorTok{$}\NormalTok{Subject}\OperatorTok{==}\DecValTok{7} \OperatorTok{|}\StringTok{ }\NormalTok{d}\OperatorTok{$}\NormalTok{Subject}\OperatorTok{==}\DecValTok{14}\NormalTok{]=}\DecValTok{714}
\CommentTok{# 8=24}
\NormalTok{d}\OperatorTok{$}\NormalTok{Subject[d}\OperatorTok{$}\NormalTok{Subject}\OperatorTok{==}\DecValTok{8} \OperatorTok{|}\StringTok{ }\NormalTok{d}\OperatorTok{$}\NormalTok{Subject}\OperatorTok{==}\DecValTok{24}\NormalTok{]=}\DecValTok{824}
\CommentTok{# 9=15}
\NormalTok{d}\OperatorTok{$}\NormalTok{Subject[d}\OperatorTok{$}\NormalTok{Subject}\OperatorTok{==}\DecValTok{9} \OperatorTok{|}\StringTok{ }\NormalTok{d}\OperatorTok{$}\NormalTok{Subject}\OperatorTok{==}\DecValTok{15}\NormalTok{]=}\DecValTok{915}
\CommentTok{# 10=17}
\NormalTok{d}\OperatorTok{$}\NormalTok{Subject[d}\OperatorTok{$}\NormalTok{Subject}\OperatorTok{==}\DecValTok{10} \OperatorTok{|}\StringTok{ }\NormalTok{d}\OperatorTok{$}\NormalTok{Subject}\OperatorTok{==}\DecValTok{17}\NormalTok{]=}\DecValTok{1017}
\CommentTok{# 11=39}
\NormalTok{d}\OperatorTok{$}\NormalTok{Subject[d}\OperatorTok{$}\NormalTok{Subject}\OperatorTok{==}\DecValTok{11} \OperatorTok{|}\StringTok{ }\NormalTok{d}\OperatorTok{$}\NormalTok{Subject}\OperatorTok{==}\DecValTok{39}\NormalTok{]=}\DecValTok{1139}
\CommentTok{# 18=23}
\NormalTok{d}\OperatorTok{$}\NormalTok{Subject[d}\OperatorTok{$}\NormalTok{Subject}\OperatorTok{==}\DecValTok{18} \OperatorTok{|}\StringTok{ }\NormalTok{d}\OperatorTok{$}\NormalTok{Subject}\OperatorTok{==}\DecValTok{23}\NormalTok{]=}\DecValTok{1823}
\CommentTok{# 22=36}
\NormalTok{d}\OperatorTok{$}\NormalTok{Subject[d}\OperatorTok{$}\NormalTok{Subject}\OperatorTok{==}\DecValTok{22} \OperatorTok{|}\StringTok{ }\NormalTok{d}\OperatorTok{$}\NormalTok{Subject}\OperatorTok{==}\DecValTok{36}\NormalTok{]=}\DecValTok{2236}
\CommentTok{# 26=49}
\NormalTok{d}\OperatorTok{$}\NormalTok{Subject[d}\OperatorTok{$}\NormalTok{Subject}\OperatorTok{==}\DecValTok{26} \OperatorTok{|}\StringTok{ }\NormalTok{d}\OperatorTok{$}\NormalTok{Subject}\OperatorTok{==}\DecValTok{49}\NormalTok{]=}\DecValTok{2649}
\CommentTok{# 27=31}
\NormalTok{d}\OperatorTok{$}\NormalTok{Subject[d}\OperatorTok{$}\NormalTok{Subject}\OperatorTok{==}\DecValTok{27} \OperatorTok{|}\StringTok{ }\NormalTok{d}\OperatorTok{$}\NormalTok{Subject}\OperatorTok{==}\DecValTok{31}\NormalTok{]=}\DecValTok{2731}
\CommentTok{# 28=60}
\NormalTok{d}\OperatorTok{$}\NormalTok{Subject[d}\OperatorTok{$}\NormalTok{Subject}\OperatorTok{==}\DecValTok{28} \OperatorTok{|}\StringTok{ }\NormalTok{d}\OperatorTok{$}\NormalTok{Subject}\OperatorTok{==}\DecValTok{60}\NormalTok{]=}\DecValTok{2860}
\CommentTok{# 30=42}
\NormalTok{d}\OperatorTok{$}\NormalTok{Subject[d}\OperatorTok{$}\NormalTok{Subject}\OperatorTok{==}\DecValTok{30} \OperatorTok{|}\StringTok{ }\NormalTok{d}\OperatorTok{$}\NormalTok{Subject}\OperatorTok{==}\DecValTok{42}\NormalTok{]=}\DecValTok{3042}
\CommentTok{# 33=54}
\NormalTok{d}\OperatorTok{$}\NormalTok{Subject[d}\OperatorTok{$}\NormalTok{Subject}\OperatorTok{==}\DecValTok{33} \OperatorTok{|}\StringTok{ }\NormalTok{d}\OperatorTok{$}\NormalTok{Subject}\OperatorTok{==}\DecValTok{54}\NormalTok{]=}\DecValTok{3354}
\CommentTok{# 34=63}
\NormalTok{d}\OperatorTok{$}\NormalTok{Subject[d}\OperatorTok{$}\NormalTok{Subject}\OperatorTok{==}\DecValTok{34} \OperatorTok{|}\StringTok{ }\NormalTok{d}\OperatorTok{$}\NormalTok{Subject}\OperatorTok{==}\DecValTok{63}\NormalTok{]=}\DecValTok{3463}
\CommentTok{# 35=40}
\NormalTok{d}\OperatorTok{$}\NormalTok{Subject[d}\OperatorTok{$}\NormalTok{Subject}\OperatorTok{==}\DecValTok{35} \OperatorTok{|}\StringTok{ }\NormalTok{d}\OperatorTok{$}\NormalTok{Subject}\OperatorTok{==}\DecValTok{40}\NormalTok{]=}\DecValTok{3540}
\CommentTok{# 37=46}
\NormalTok{d}\OperatorTok{$}\NormalTok{Subject[d}\OperatorTok{$}\NormalTok{Subject}\OperatorTok{==}\DecValTok{37} \OperatorTok{|}\StringTok{ }\NormalTok{d}\OperatorTok{$}\NormalTok{Subject}\OperatorTok{==}\DecValTok{46}\NormalTok{]=}\DecValTok{3746}
\CommentTok{# 38=48}
\NormalTok{d}\OperatorTok{$}\NormalTok{Subject[d}\OperatorTok{$}\NormalTok{Subject}\OperatorTok{==}\DecValTok{38} \OperatorTok{|}\StringTok{ }\NormalTok{d}\OperatorTok{$}\NormalTok{Subject}\OperatorTok{==}\DecValTok{48}\NormalTok{]=}\DecValTok{3848}
\CommentTok{# 41=65}
\NormalTok{d}\OperatorTok{$}\NormalTok{Subject[d}\OperatorTok{$}\NormalTok{Subject}\OperatorTok{==}\DecValTok{41} \OperatorTok{|}\StringTok{ }\NormalTok{d}\OperatorTok{$}\NormalTok{Subject}\OperatorTok{==}\DecValTok{65}\NormalTok{]=}\DecValTok{4165}
\CommentTok{# 43=59}
\NormalTok{d}\OperatorTok{$}\NormalTok{Subject[d}\OperatorTok{$}\NormalTok{Subject}\OperatorTok{==}\DecValTok{43} \OperatorTok{|}\StringTok{ }\NormalTok{d}\OperatorTok{$}\NormalTok{Subject}\OperatorTok{==}\DecValTok{59}\NormalTok{]=}\DecValTok{4359}
\CommentTok{# 44=75}
\NormalTok{d}\OperatorTok{$}\NormalTok{Subject[d}\OperatorTok{$}\NormalTok{Subject}\OperatorTok{==}\DecValTok{44} \OperatorTok{|}\StringTok{ }\NormalTok{d}\OperatorTok{$}\NormalTok{Subject}\OperatorTok{==}\DecValTok{75}\NormalTok{]=}\DecValTok{4475}
\CommentTok{# 45=52}
\NormalTok{d}\OperatorTok{$}\NormalTok{Subject[d}\OperatorTok{$}\NormalTok{Subject}\OperatorTok{==}\DecValTok{45} \OperatorTok{|}\StringTok{ }\NormalTok{d}\OperatorTok{$}\NormalTok{Subject}\OperatorTok{==}\DecValTok{52}\NormalTok{]=}\DecValTok{4552}
\CommentTok{# 47=78}
\NormalTok{d}\OperatorTok{$}\NormalTok{Subject[d}\OperatorTok{$}\NormalTok{Subject}\OperatorTok{==}\DecValTok{47} \OperatorTok{|}\StringTok{ }\NormalTok{d}\OperatorTok{$}\NormalTok{Subject}\OperatorTok{==}\DecValTok{78}\NormalTok{]=}\DecValTok{4778}
\CommentTok{# 50=57}
\NormalTok{d}\OperatorTok{$}\NormalTok{Subject[d}\OperatorTok{$}\NormalTok{Subject}\OperatorTok{==}\DecValTok{50} \OperatorTok{|}\StringTok{ }\NormalTok{d}\OperatorTok{$}\NormalTok{Subject}\OperatorTok{==}\DecValTok{57}\NormalTok{]=}\DecValTok{5057}
\CommentTok{# 51=69}
\NormalTok{d}\OperatorTok{$}\NormalTok{Subject[d}\OperatorTok{$}\NormalTok{Subject}\OperatorTok{==}\DecValTok{51} \OperatorTok{|}\StringTok{ }\NormalTok{d}\OperatorTok{$}\NormalTok{Subject}\OperatorTok{==}\DecValTok{69}\NormalTok{]=}\DecValTok{5169}
\CommentTok{# 53=55}
\NormalTok{d}\OperatorTok{$}\NormalTok{Subject[d}\OperatorTok{$}\NormalTok{Subject}\OperatorTok{==}\DecValTok{53} \OperatorTok{|}\StringTok{ }\NormalTok{d}\OperatorTok{$}\NormalTok{Subject}\OperatorTok{==}\DecValTok{55}\NormalTok{]=}\DecValTok{5355}
\CommentTok{# 56=66}
\NormalTok{d}\OperatorTok{$}\NormalTok{Subject[d}\OperatorTok{$}\NormalTok{Subject}\OperatorTok{==}\DecValTok{56} \OperatorTok{|}\StringTok{ }\NormalTok{d}\OperatorTok{$}\NormalTok{Subject}\OperatorTok{==}\DecValTok{66}\NormalTok{]=}\DecValTok{5666}
\CommentTok{# 58=67}
\NormalTok{d}\OperatorTok{$}\NormalTok{Subject[d}\OperatorTok{$}\NormalTok{Subject}\OperatorTok{==}\DecValTok{58} \OperatorTok{|}\StringTok{ }\NormalTok{d}\OperatorTok{$}\NormalTok{Subject}\OperatorTok{==}\DecValTok{67}\NormalTok{]=}\DecValTok{5867}
\CommentTok{# 61=64}
\NormalTok{d}\OperatorTok{$}\NormalTok{Subject[d}\OperatorTok{$}\NormalTok{Subject}\OperatorTok{==}\DecValTok{61} \OperatorTok{|}\StringTok{ }\NormalTok{d}\OperatorTok{$}\NormalTok{Subject}\OperatorTok{==}\DecValTok{64}\NormalTok{]=}\DecValTok{6164}
\CommentTok{# 62=71}
\NormalTok{d}\OperatorTok{$}\NormalTok{Subject[d}\OperatorTok{$}\NormalTok{Subject}\OperatorTok{==}\DecValTok{62} \OperatorTok{|}\StringTok{ }\NormalTok{d}\OperatorTok{$}\NormalTok{Subject}\OperatorTok{==}\DecValTok{71}\NormalTok{]=}\DecValTok{6271}
\CommentTok{# 68=74}
\NormalTok{d}\OperatorTok{$}\NormalTok{Subject[d}\OperatorTok{$}\NormalTok{Subject}\OperatorTok{==}\DecValTok{68} \OperatorTok{|}\StringTok{ }\NormalTok{d}\OperatorTok{$}\NormalTok{Subject}\OperatorTok{==}\DecValTok{74}\NormalTok{]=}\DecValTok{6874}
\CommentTok{# 70=72}
\NormalTok{d}\OperatorTok{$}\NormalTok{Subject[d}\OperatorTok{$}\NormalTok{Subject}\OperatorTok{==}\DecValTok{70} \OperatorTok{|}\StringTok{ }\NormalTok{d}\OperatorTok{$}\NormalTok{Subject}\OperatorTok{==}\DecValTok{72}\NormalTok{]=}\DecValTok{7072}
\CommentTok{# 73=76}
\NormalTok{d}\OperatorTok{$}\NormalTok{Subject[d}\OperatorTok{$}\NormalTok{Subject}\OperatorTok{==}\DecValTok{73} \OperatorTok{|}\StringTok{ }\NormalTok{d}\OperatorTok{$}\NormalTok{Subject}\OperatorTok{==}\DecValTok{76}\NormalTok{]=}\DecValTok{7376}
\end{Highlighting}
\end{Shaded}

While this code solved the immediate problem, it was a clunky solution
that could have introduced many new errors that would be difficult to
detect. What if the student lost his notes indicating which participant
ids should be paired before running the analysis? What if those notes
contained discrepancies he could not sort out (e.g., writing down the
same id accidentally for two participants)?

Furthermore, the inelegant coding solution needed to manage this is also
prone to human error. Mis-typing an id number could mean creation of a
bogus pairing, accidentally assigning three sessions to one person, etc.
After doing this, we had to check carefully that all resulting ids had
the right number of sessions, trials, and conditions.

In the end, this worked out fine. But work and hassle would have been
avoided if each participant were given a consistent id to start with, or
alternatively if an id system was used that resulted in unique session
ids that were related in some regular way. This problem would of course
have been compounded if the design had been more complicated (e.g., what
if there were \textgreater{} 2 sessions?) and as the sample size
increased (imagine sorting this for 150 participants instead of
\textasciitilde{}30).

\subsection{Ids are always arbitrary,
anonymous}\label{ids-are-always-arbitrary-anonymous}

The system you choose for assigning participant numbers should not
preserve any links between the participant's identity in real-life and
their identity as a unique contributor to your data set. Ids should not
be derived from any university-linked identifier, address, or birthdate.
They should not include the date or time of the session. Typical ways to
assign ids are to choose a starting number and increment upward with
each new participant.

If you are running a multi-experiment project, it is a good idea to
segregate the participant ids of subsequent experiments in some way that
makes them obviously distinct from previous experiments in the sequence
(e.g., experiment 1 starts from 100, experiment 2 from 200, etc.). This
way, if you ever combine data to perform a multi-experiment analysis,
you still have unique participant ids.

\hypertarget{ethics-approvals-currently-in-place}{\section{Ethics
approvals currently in
place}\label{ethics-approvals-currently-in-place}}

\subsection{Limits of memory
(EC.17.09.12.4952G)}\label{limits-of-memory-ec.17.09.12.4952g}

This is a generic approval valid until 31 August 2022. This heading
covers behavioural studies with adults (17-40 years old) in which
participants are asked to try to remember something, sometimes while
also doing something else. This heading currently applies to
single-session studies of 90 minutes or less.

A new risk assessment will be due prior to the expiration of the
project.

\subsection{Memory storage and processing across the lifespan
(EC.18.09.185339GR)}\label{memory-storage-and-processing-across-the-lifespan-ec.18.09.185339gr}

This is a generic approval valid until 30 September 2023. This heading
covers behavioural studies with adults and children, eye-tracking, and
multi-session data collection. Examples of studies under this heading
include manipulations of complex span tasks and the Flavell et al
Registered Replication Report.

A new rish assessment will be due prior to the expiration of the
project.

\chapter{Ethics at Cardiff School of
Psychology}\label{ethics-at-cardiff-school-of-psychology}

Ethical approval is an essential part of conducting research. At Cardiff
University, each academic school has its own School Research Ethics
committee (SREC) that is responsible for ensuring that all research
carried out within the School has been subject to an ethical review.The
university's procedure for ethical review allows staff and students to
achieve research excellence and produce work with the highest standards
of research integrity and professionalism.

\section{Who needs to apply?}\label{who-needs-to-apply}

Staff and students conducting research on humans must apply for ethical
approval. In some cases, ethical approval may have been granted by an
external organisation (e.g., NHS) and this approval letter must be
submitted to the SREC.

\section{How to apply for ethical
review?}\label{how-to-apply-for-ethical-review}

A great starting point for information and resources about the ethical
review process can be found at
\href{https://inside.psych.cf.ac.uk/}{InsidePsych}. Log in with your
Cardiff University credentials and navigate to the
\emph{\textbf{Ethics}} tab which can be found in the Undergraduate,
Postgraduate, and Staff pages.

\subsection{Types of applications}\label{types-of-applications}

All applications and supporting documents must be emailed to the Ethics
committee at
\href{mailto:psychethics@cardiff.ac.uk}{\nolinkurl{psychethics@cardiff.ac.uk}}

\textbf{1. Fast track applications:} This procedure is used for
proposals that do not raise any ethical issues and recruits participants
from the School's participant panel. To recruit from the panel, you will
need to sign up for a researcher account on the School of Psychology's
\href{https://cardiff.sona-systems.com/Default.aspx?ReturnUrl=/}{Experimental
Management System} and obtain ethical approval before you begin
recruiting participants. Fast track applications can also be used to
make minor amendments to a project that has been previously approved.
These applications must be submitted by midday on a Friday and a
decision will be made within 7-10 days.

\textbf{2. Full applications:} These applications require more detailed
descriptions of the experimental procedures, research design, and
ethical considerations. Full applications must be submitted at least two
weeks before the Ethics Committee's next meeting. The committee usually
meets on the first Tuesday of every month.

Full applications are used when a research project involves (a)
recruiting special groups of participants (e.g., children, people in
custody, those lacking capacity to give consent, or people engaged in
illegal activities); (b) highly sensitive topics (e.g., domestic
violence, terrorism, etc), (c) the use of a drug (e.g., tobacco,
caffeine, or alcohol), or (d) the collection or use of human tissue. In
essence, if the research poses an ethical consideration, a full
application must be submitted for review.

\textbf{3. Generic applications:} Sometimes, researchers may have a
range of studies they would like to conduct within a research paradigm.
In these cases, a generic application can be completed so long as there
is sufficient description of the different studies they wish to
undertake.

For example, a researcher may want to study the effectiveness of recall
in different situations. A generic application can be submitted so long
as the different studies within this paradigm are described, for
example, testing memory in a noisy environment or after participants
have had a cup of coffee. Depending on the ethical considerations in the
study, these applications may either be a fast track or full
application. \textbf{Undergraduate students are not permitted to apply
for generic approval.}

\textbf{4. Amendments to approved projects:} Sometimes, changes may be
made to a project that already has ethical approval. If they are changes
to the research team or the project end date then send an email to the
Ethics Committee with the details of the change and the project's ethics
reference number. All other changes require the submission of a new
ethics proforma along with the previous ethics reference number for the
project. Remember to provide sufficient detail about the changes as if
they are minor amendments they will be dealt with through the fast track
route while if they are major changes they will be considered in the
same way as a full proposal.

\section{What forms will I need to
complete?}\label{what-forms-will-i-need-to-complete}

\subsection{Ethics proforma}\label{ethics-proforma}

An ethics proforma must be completed for all applications except those
seeking approval for minor amendments. The proforma can be downloaded
\href{https://inside.psych.cf.ac.uk/}{here}. You will also find guidance
notes for the proforma on this page. The proforma asks for details about
your study - who is conducting it, who are the participants, and what
methods will be employed. If you are conducting a project with no
foreseeable ethical considerations then your application will be
considered under the fast track route and you will need to describe the
study in Box A. If you are conducting a project that raises ethical
considerations then choose Box B and write a full proposal.

\subsection{Full proposal}\label{full-proposal}

The purpose of a full proposal is to provide the Ethics committee with
sufficient information about the project to help them evaluate the
ethical implications and any detrimental impact that the study may have
on prospective participants. Make sure to number the pages in the
proposal and label appendices correctly. In addition to submitting the
proforma, your proposal must contain the following information:

\begin{itemize}
\item
  \textbf{Project overview:} Include information about the aims of the
  study, the rationale, and methodology. You must also mention the
  estimated start date and duration of the project.
\item
  \textbf{Participants:} State how many participants you intend on
  recruiting, how old they will be, and how you will recruit them.
  Remember to mention any inclusion / exclusion criteria as well. If you
  intend on using flyers or e-mails to recruit participants you can
  include these documents in the appendix. If you are recruiting
  children from schools then include a Gatekeeper letter which will be
  used to contact head teachers when requesting their participation
  (samples can be downloaded from the Ethics section of
  \href{https://inside.psych.cf.ac.uk/}{InsidePsych}).
\item
  \textbf{Materials and procedure:} Highlight what materials will be
  used (e.g., a computer and keyboard), the general procedure during
  data collection (i.e., obtaining informed consent, conducting the
  experimental task or questionnaire, and debriefing), and procedures
  for storage of personal and experimental data. Again, include related
  documents (e.g., questionnaires, information sheets, etc) as
  appendices.
\item
  \textbf{Ethical considerations:} A brief description of the ethical
  considerations associated with the study along with a proposed plan of
  action to reduce their impact.
\end{itemize}

\subsection{Risk assessment}\label{risk-assessment}

Risk assessments are a mandatory part of the ethical review process.
When you fill in the proforma, you will also be required to complete an
online risk assessment form (link available within the proforma). This
form is designed to provide an overall estimate of the risk involved in
the project and who is likely to be affected. You will be asked to rate,
on a scale of 1 to 5, the likelihood of a hazard causing a problem and
the severity of injury resulting from the hazard. For example, equipment
like computers may pose a risk of electric shock, however, with control
measures like PAT testing the likelihood of electric shocks may be low
(rated as 1 - very unlikely) while the severity may be rated as 3 to
indicate that an electric shock may lead to some temporary injuries.

Once you have completed and submitted the form, you will receive an
email to indicate if a full risk assessment is needed. If so, the email
will also contain a link for the full risk assessment form, guidance
notes, and details for where to send the completed form. You will have
to give information about the hazard, measures to mitigate its impact,
and ratings for the likelihood and severity of the hazard.
\textbf{Please note you cannot submit your ethics application to the
committee without a complete risk assessment.} When planning an
application, assume that you will be asked to revise your risk
assessment based on feedback, which may take a few extra days. Once your
risk assessment is approved, you will receive a unique \emph{Risk
assessment receipt number} via email. You must include this receipt
number in your Ethics Proforma when you submit (your application will be
returned to you if this information is not included).

You can now submit your ethics proforma together with the proposal (if
required) and any supporting documents for ethical review!
\textbf{Remember, you CANNOT begin any research activities until your
application has been approved by the Ethics Committee.}

\chapter{Data security}\label{data-security}

Data security refers to the principles followed to protect our data.
There are two sides to this: ensuring that our data are not lost and
ensuring that aspects of our data that are personal are kept
confidential.

All lab personnel should complete the
\href{https://intranet.cardiff.ac.uk/staff/news/view/211993-information-security-training-when-will-you-complete-yours}{web-based
Data Security training module} offered by Cardiff University.

The data we post publicly online should not afford participants'
identities to be discovered. There nonetheless remains some information
in the data file that interested parties could use in conjunction with
other known information to identify a participant. Suppose you are a
private investigator, and your client wants you to spy on her child at
university. You know your mark is enrolled on the Psychology degree, and
you find out that completing this degree involves participating in
laboratory studies. You think these studies might be a source of key
information about how well your mark can remember abstract visual
images, so you follow him to the laboratory, noting the date and time he
participated. Once the data are posted online, you can download them and
find the participant whose data were recorded on that date and time,
even if no other information in the data uniquely identifies your mark.
Knowing what we know about the distortibility of data, we can take those
visual recognition task data and dubiously infer all sorts of
maliciousness.

This is obviously ridiculous, but technically possible. It is referred
to as the \emph{motivated intruder test}, and we cannot consider our
data truly anonymous if a motivated intruder could easily uncover and
exploit the indirect links between anonymous data and identifiable
information we may have access to. To comply with EU privacy
regulations, we need to take steps to protect participant identities
even when we think our data are harmless. This holds regardless of
whether we plan to make the anonymous data set publicly available.
Unless you are storing the data only on secured Cardiff University
resources, it must be properly anonymized. This means that before we
anonymize it, data cannot be held even temporarily on personal disks, on
DropBox, or Google Docs, and cannot be emailed to non-university email
addresses.

\section{Backing up digital data}\label{backing-up-digital-data}

\subsection{During data collection}\label{during-data-collection}

All lab personnel will be given access to a OneDrive folder
``labBackUp\_cmorey''. While data collection is ongoing, the researcher
responsible for data collection should add newly acquired data to the
OneDrive lab back-up every day new data are acquired. There are
sub-directories in \emph{labBackUp\_cmorey} for all the physical spaces
we use to collect data. Simply copy the your experiment directory on the
physical machine to the corresponding sub-directory in
\emph{labBackUp\_cmorey}.

Note that here it will be necessary to make sure that your experimental
directory has a unique name. The short-hand name you use for your
project can be used for this. Do not choose names like ``project'',
``dissertation'', ``thesis'', ``experiment 1''.

These data are \emph{confidential} because they include timestamps that,
if linked to the participant's EMS schedule or to information the
participant discloses publicly about their participation in experiments,
their identity could be reconstructed. As such, you should not save
these data onto personal computers or drives.

\subsection{When data collection is
closed}\label{when-data-collection-is-closed}

Once we have decided not to collect any more data on a project, it is
time to compile and anonymize that data set. This should always occur
before any analyses are attempted, and must always be done in a scripted
fashion. For lab members not fluent in scripting, this should be done in
a face-to-face meeting with the PI or another senior lab member. This is
the \textbf{Data Round-up Meeting}.

Everyone involved in data collection should attend the \textbf{Data
Round-up Meeting}, because we will at the same time make decisions about
participant exclusions. Everyone who collected data should bring their
lab notebook. If a face-to-face meeting including all personnel is
impossible, data collectors should at least make their lab notebook
available to the PI for this meeting, and have prepared a summary of
which participants they think must be left out and why.

First, the team shall compile the data associated with the project from
the \emph{labBackUp\_cmorey} folder. It is essential that this folder is
up-to-date ahead of the meeting - data collectors, please ensure that
you have copied all data to the back up ahead of time.

At the meeting, we will use this code to compile each individual
backed-up data file into a single data frame, \emph{d}:

\begin{Shaded}
\begin{Highlighting}[]
\NormalTok{d =}\StringTok{ }\KeywordTok{map_df}\NormalTok{(}\KeywordTok{list.files}\NormalTok{(}\DataTypeTok{path =} \StringTok{""}\NormalTok{, }\DataTypeTok{full.names =} \OtherTok{TRUE}\NormalTok{), }
\NormalTok{                 read_csv, }\DataTypeTok{col_names =}\NormalTok{ T)}
\end{Highlighting}
\end{Shaded}

This code creates a dataframe called \emph{d} that includes all the .csv
files in the project directory, which will be entered into the
\emph{path} argument.

We shall then examine \emph{d}'s headings for information that could
identify the participant. Anything that has to do with the date and time
of the session must be omitted. Recorded birthdates must be omitted.
Usually, we would not have recorded anything more personal than that,
but if so, it must be omitted at this stage.

Obviously, we do need the participants' ages to describe our sample. We
can calculate that. First, it is a good idea to first look at the
entries for any year-of-birth or birthdate variable to make sure they
are plausible.

\begin{Shaded}
\begin{Highlighting}[]
\KeywordTok{unique}\NormalTok{(d}\OperatorTok{$}\NormalTok{YoB)}
\end{Highlighting}
\end{Shaded}

This code will produce a list of each unique value at YoB in the data
frame. Check that they all seem plausible. If they are not, consider
whether a weird value might have been a typo by figuring out who
interacted with that participant and consulting the relevant notebook.
The people in the room must reach an agreement on what to do in the
event of a weird value. Options include: considering the age missing for
that participant, omitting the participant if they fall outside the
recruited age range, judging that a transposition was entered (e.g.,
entering a YoB of 2020 rather than 2002 was an error because the
participant was definitely an adult, or that a 2-digit value like 98
actually meant 1998). In cases where the typo seems obvious, we can
decide to fix it, but this must be documented in the script.

If we recorded \emph{Year of Birth}, we can now create a new variable
converting birth year to age at the time of the study. For most
purposes, this does not need to be very precise. We can do this with a
simple \emph{mutate}.

\begin{Shaded}
\begin{Highlighting}[]
\NormalTok{d =}\StringTok{ }\NormalTok{d }\OperatorTok\StringTok{ }\KeywordTok{mutate}\NormalTok{(}\DataTypeTok{age =} \DecValTok{2018} \OperatorTok{-}\StringTok{ }\NormalTok{YoB) }\CommentTok{# current year - column name containing the birth year}
\end{Highlighting}
\end{Shaded}

If we recorded the full birthdate, we probably need age in more
precision. The function \emph{lubridate} is useful for sorting out
birthdates and calculating precise ages from birthdates and run dates.

\begin{Shaded}
\begin{Highlighting}[]
\KeywordTok{library}\NormalTok{ (lubridate)}
\CommentTok{# lubridate example here}
\end{Highlighting}
\end{Shaded}

Once we have created whatever new, anonymous variables we need from the
potentially identifying data, we can create a new data set with the
identifiable data omitted. We would do that with script like this:

\begin{Shaded}
\begin{Highlighting}[]
\CommentTok{# Remove columns that contain ids or timestamps (times could limit the pool to specific people given that the location of the test is known)}
\NormalTok{symsp =}\StringTok{ }\NormalTok{symsp }\OperatorTok\StringTok{ }\KeywordTok{select}\NormalTok{(}\OperatorTok{-}\KeywordTok{c}\NormalTok{(Clock.Information, Email, SessionDate, SessionTime)) }\CommentTok{# The "-" negates the select, meaning that everything *except* the named columns remain.}
\end{Highlighting}
\end{Shaded}

Include all the variables you want to omit within the -c() list. The
resulting data frame will include everything except those variables.
View the resulting data frame to check that this indeed worked.

This new data frame is the one we will analyze. Save it now to the PI's
local directory, and also upload it to the OSF page for this project so
all of the lab personnel who need to use it can access it.

\begin{Shaded}
\begin{Highlighting}[]
\CommentTok{# Save it - this is the version that goes on OSF}
\KeywordTok{write_csv}\NormalTok{(symsp, }\StringTok{"symSpan_RuG_anon.csv"}\NormalTok{)}
\end{Highlighting}
\end{Shaded}

We may occasionally need the identifiable data, particularly the
timestamps in it. If we need to prove that the data were collected after
a certain date, these timestamps are strong evidence. But the original
data need to be kept only in a secure place that is never accessible
publicly and that is regularly automatically backed up. Currently the
only available spaces that meet this requirement are the Shared and Home
drives. Our policy: the compiled, non-anonymized data are stored on
Candice Morey's Home drive. She will store the script we write to
anonymize the data and the original data on a sub-directory of My
Documents/dataSets\_cardiff on her Home Drive.

\subsection{Cleaning up lab
directories}\label{cleaning-up-lab-directories}

After we have compiled the data, created anonymized versions, and backed
up the original data to Candice's H drive, it should not matter whether
the original data files stored on \emph{labBackUp\_cmorey} or on local
machines are lost. These sources are sufficiently secure for our data,
which are at most \emph{confidential}, never \emph{highly confidential}.
In the event that we collect \emph{highly confidential} data, such data
should be deleted from local machines and OneDrive as soon as we have
the compiled data backed up to the more secure H drive. For data that is
\emph{confidential} we need not rush to delete it. We shall allow most
data to exist on \emph{labBackUp\_cmorey} until the space is required
for newer data, and then delete the oldest files that have been compiled
and backed up first. The researcher responsible for the data should
delete it from the local machine where it was collected after data
round-up for that project is finished.

\subsection{Summary}\label{summary}

Individual data files may be found on the local machine on which they
were collected, unless there has been a crash or overwriting error.
Assuming our lab back up procedure was followed, individual data files
should be redundantly available on our shared OneDrive back-up.

Once we have compiled and anonymized the data, it will not be necessary
to access the individual data files. Lab personnel associated with a
project shall all have access to the anonymized data via our shared OSF
project page. This anonymized data file may be saved redundantly
anywhere. It will not contain sensitive information.

If you somehow lose the anonymized data file and it is somehow gone from
OSF, we can recover it by re-running the anonymization script we created
on the original compiled data. The PI controls access to the original
data, so requests for recovery from the original data must go through
her. If we need the original data to prove something about the time
period of data collection, the PI must also arrange access.

\section{Processing paperwork}\label{processing-paperwork}

\subsection{Consent forms}\label{consent-forms}

Hopefully, the only paper associated with your project is the signed
consent forms. These are definitely identifiable - they contain the
participants' names and dates they took part. We must keep copies of
these because it is necessary that we be able to prove that a particular
person consented to take part in our research, if the person ever
claimed s/he did not give consent.

Once data collection on your project is closed, you must turn these
papers over to the PI. First, scan the consent forms into one .pdf
document. Enter the PI's email address as recipient. Make sure this is
done before the compilation meeting. During the compilation meeting, we
shall check that the number of pages in the .pdf consent scan matches
the number of participants in the study. If not, we may need to discover
which participant did not consent, and remove that participant's data
using the timestamp information. Candice will save the consents .pdf to
her H drive. The paper consents will be destroyed in the next
confidential waste pick-up.

The consents sub-directory on Candice's H-drive is organized by year. We
are required to keep the signed consent forms for 7 years. Annually, she
shall delete the sub-directories more than 7 years old.

\subsection{Paper data}\label{paper-data}

If your study involves data collection on paper forms, you must back
these up digitally by scanning them to .pdf as soon as your data
collection closes. While collecting data, you should leave the paper
forms in an organized space within the lab.

Paper data should also exclude personally identifiable information. It
will also be considered \emph{confidential} because some combination of
lab personnel could work out the identity of the participant by linking
the participant id on the paperwork with corresponding digial data with
a timestamp. This is why paper data should remain in the secure lab.
Once scanned, the digital file shall be saved to the appropriate
OneDrive directory and to Candice Morey's H drive.

Data requiring transcription or coding must be duplicated by two
independent coders, with any disgreements resolved by a third person.
For projects involving coding or transcription, once we have multiple
ratings the raters and PI (or a senior researcher) shall meet to compare
the ratings and arbitrate any differences. In this meeting, the two
rater's transcriptions shall be merged, a resulting measure evaluated,
and ultimately the resulting agreed measure shall be merged with the
other project data. All steps will be documented via script. Once this
has been completed, and the data have been uploaded to OSF (for
anonymized data) or Candice's H-drive (for nonanoymous data), the paper
forms shall be destroyed via Confidential Waste collection.

\section{Confidentiality}\label{confidentiality}

We have procedures built into our back-up process to prevent
potentially-identifiable research data from leaving the secure
univerisity environment. However, our biggest efforts for solving this
problem are taken during data collection and study design. Whenever
possible (which is nearly always), we do not record any identifiable
information about our participants. Unless you need precise information
about their ages at test, we do not record birth dates. We never record
proper names or any stable identification number with experimental data.
When we need to link a participant's data across multiple sessions, we
will do this with a key of real-life identity and contact information to
participant numbers, which is destroyed once we have finished collecting
data. When we need to use such a key, it is maintained by one
researcher, and stored within the secured university environment.

Our ethics approval states that we shall destroy this key when data
collection is finished. For projects involving a key, key destruction
will be the last item on the agenda of the \textbf{Data Round-up
Meeting}. Once we have compiled and written the anonymous data files, we
shall run tests to be sure that we have correctly matched up participant
ids across sessions. Once satisfied of this, we shall delete all copies
of the key.

\chapter{Designing your study}\label{designing-your-study}

Investing time into designing your study pays off in the long run.
Testing participants is tedious, hard work. During some periods and for
some kinds of participants, it is expensive. You want to be as sure as
possible that the experiments you run are capable of addressing the
points that you are trying to address. What's more, you want your
experiments to address these points thoroughly and well.

We always try to design experiments according to the following
principles, unless we have a very good reason to deviate from them.

\section{Use computers whenever possible to control stimulus
presentation and response
collection}\label{use-computers-whenever-possible-to-control-stimulus-presentation-and-response-collection}

Back in the late 1990s when I was an undergraduate student interested in
cognitive psychology, I worked in several labs trying to figure out what
kind of research I wanted to specialize in. One thing became completely
clear with experience, and it is even more true now than it was then: if
you are serious about becoming a cognitive psychologist, you must be
proficient at getting computers to work for you. Without decent tech
skills, your work will always be sub-par. Yes, some classic papers were
written in which the researchers presented their stimuli in low-tech
ways, and the quality of the results were based on their great ideas and
committment to conscientiousness. But see, even if you have a comparable
commitment to conscientiousness and comparably great ideas, there will
be someone else who has those things and also has sufficiently good
technical skills to automate their research designs. And in 99 cases out
of 100, that automation will improve the project. Why is that?

\subsection{People make more mistakes}\label{people-make-more-mistakes}

Suppose you want to show participants a stimulus for 1 second. For our
research, it may not be extremely important whether the participant is
exposed to the stimulus for 900 ms versus 1200 vs vs 1000 ms, the
fluctuations in time you might expect if a human is reading out stimuli
or manually displaying them on flashcards. But you don't really know, do
you? Suppose a reviewer asks for some confirmation about timing, or
thinks that a particular pattern in the data might make sense for
stimuli presented for 1100 ms, but would be surprising if they were
presented for \textless{} 1000 ms. You will need to be reasonably
certain about major elements of your methods, and reasonable certainty
won't be possible with human-manual presentation.

Computers have brief hiccups that can cause events to be displaced from
their stated timeline slightly. These displacements are usually on the
order of 7 ms, and these blips are recorded. People deviate way more.
Suppose you are showing a participant stimuli on a series of flashcards,
aiming for exposures of 1 second each, and you need to sneeze. There is
going to be an irregularity of way more than 7 ms. What's more, you will
have to decide, in real-time, whether to record the presence of that
regularity in the trial somewhere, which will again take more than 7 ms.
If you decide not to, then we will not know from the data that the
irregularity occurred, and that trial may be analyzed as though it were
normal. There isn't a great solution for this. But with a good
experimental presentation program, we would always be able to check
whether our assumptions about timing were met, and if not, how much
deviation there was. We could quickly exclude any trial with massive
irregularities if we decided it was important to do so.

Another example: people are notoriously terrible at judging randomness.
If we want some variable to be presented in random order, a human will
have trouble distributing the levels of that variable in a random way.
What's more, different humans will have somewhat different ideas about
which patterns look random and which don't. It's better to let machines
sort this out, based on some rules we restrict them with.

\subsection{People are biased}\label{people-are-biased}

The first rule of objective research is that your hypothesis should not
be allowed to affect the outcome of a test. This becomes impossible to
regulate when the human delivering the stimuli and recording the
responses has a hypothesis about how participants will perform under
various conditions. Effects of experimenter bias on test outcomes have
been demonstrated in many settings.
\href{https://en.wikipedia.org/wiki/Tryon\%27s_Rat_Experiment}{An
example}: Tryon ran a research program in the 1930s and 1940s aiming to
test genetic inheritance of intelligence in rats. He tested rats on
completing a maze, and segregated the rats into groups based on how well
they got through the maze. He let the ones with the fewest errors
interbreed and the ones with the most errors interbreed, and tested
their offspring on mazes for generations. The rats in the ``bright''
population got better and better than the rats in the ``dull''
populations generations after the first selection occurred.

This may not be so surprising: nearly everyone believes that
intelligence is at least partially genetic. But there was more to
Tryon's measure than just that smart rats breeding with other smart rats
produces smart baby rats. Tryon's studies were not blind. The
researchers working with the rats knew which ones were ``bright'' and
which were ``dull''. Rosenthal and Kode (1963) replicated the research
design, but randomly assigned rats from the same population to
``bright'' and ``dull'' groups. There were in fact no systematic
differences between their ``bright'' and ``dull'' rats, except that the
researchers working with them believed that there were. Rosenthal and
Kode found the same effects as Tryon: the ``bright'' rats produced
offspring better at completing mazes than the ``dull'' rats.

One can imagine how this played out, and the many sources where bias
that might affect the rats behavior itself or even just the recordings
could be introduced. The researchers would have cared for the rats,
making sure they had food, water, stimulation, and clean bedding. The
researchers would have put the rats in the maze, counted their errors
and timed their performance. The ``bright'' rats might have been petted
more, had their food and water refilled more often or more quickly. When
being put in the maze, there could have been differences in judging what
was an error, or systematic differences in how quickly the stop watch
was triggered and stopped. If you expect the ``bright'' rat to go very
fast, perhaps you very attentive to its run, and extra vigilant about
stopping the timer.

This bias isn't restricted to rats. It can happen any time a person has
to make a judgment, however minor, about what a participant did, or make
a decision about when to deliver a stimulus, stop a clock, record an
error, etc. Can this be counteracted? We could take care to always
ensure data collectors do not know the hypothesis. But then who would
collect data? Usually, students collect data for their own project,
which they have contributed to developing. They can hardly keep
themselves from forming expectations about the results. The best and
most practical way to restrict experimenter bias is to take as many
decisions as possible out of the hands of the human experimenters during
data collection.

\subsection{\texorpdfstring{It's more work in the end \emph{not} to
automate}{It's more work in the end not to automate}}\label{its-more-work-in-the-end-not-to-automate}

All of our data will eventually need to be quantitatively analyzed. Even
data that may begin as qualitative, like responses to questions about
how participants performed a task, would eventually need to be coded as
falling into one response category or another for some kind of
quantitative analysis. So, if you plan to have participants manually
write responses, or if you code their spoken responses in real time by
hand, you will still need to transfer all of that data to a .csv file.
Errors will certainly be introduced when transcribing, so it would not
be enough for one person to transcribe these data: it would need to be
done independently by two people, compared, and then discrepancies would
need to be transcribed again.

Properly programmed and tested, the computer cannot mistake which button
the participant pressed, or where the participant clicked the mouse, or
what the participant typed. The computer's ``judgments'' need to be
checked at the start, to be sure we know it is systematically recording
what we meant for it to record, but once we are satisfied, we can be
confident that it is recording these responses the same way every time.
It will not get tired and make random errors. It will not forget which
button is which. So the responses it records do not usually need the
time-consuming checking that human-coded data need. A little time spent
upfront will therefore save you hours of painstaking work later.

\section{Work from a good example}\label{work-from-a-good-example}

Whether you are a skilled programmer or totally new to it, your first
step to programming your study should be to find an example program that
does part of what you need, and work from that. Even though we are
currently switching from closed-source E-Prime to open-source PsychoPy,
we already have programs or programs-in-progress that execute spatial
and verbal memory span tasks, display spatial locations around a
circular perimeter, and administer complex span. Don't start from
nothing: break down what you need for your program to do, and ask other
lab members for advice finding examples.

Examples can come from inside or outside the lab, but if you are
programming something that is incrementally different from something we
already have, you should work from the lab version. This is important
because we want to make sure that different iterations of our studies
differ only in the ways we are aware of. If, when exploring programs
from other labs, you discover a better way of implementing something,
put that on the agenda for the next lab meeting, and explain what we
would gain from adopting the other method. Then 1) we all learn
something new, 2) we decide as a group whether the method you discovered
is worth implementing across projects, 3) we are all at least aware if
one task iteration is going to differ in implementation from other
similar implementations.

Another reason to work from an in-lab example when implementing your
project is that it increases the likelihood that programming conventions
some of us may have previously agreed upon persist in new programs. It
is always rough figuring out just what someone else's code does. The
platforms we favor have some features that help with this, and by
working from an in-lab example, you may be able to continue in the style
that the rest of us understand.

Most importantly, we don't have a formal ban on closed-source software.
We may sometimes use E-Prime, MatLab, or whatever when it is convenient
to do so. Good reasons for this include: carrying on from existing
projects that use those platforms (e.g., avoiding ``changing horses
mid-stream''), or when we are collaborating with colleagues who insist
on these (though we will generally try to persuade them to let us
provide open-source materials). If you are starting a new project,
assume that you will be using open-source software. We do not want to
use closed-source software indefinitely, and starting anything new with
it perpetuates its use. Consider (especially if you are a student) that
your next job may not include access to E-Prime or MatLab, so learning
an open-source solution makes it more likely that you can continue
working with the platform you become familiar with later, rather than
being forced by a new employer to learn something else.

We favor open-source programming platforms that make use of some sort of
graphical user interface, like PsychoPy and OpenSesame. Even if you are
an expert coder, you should plan to implement as such as possible using
the GUI framework. This is because our lab will never exclusively
feature expert coders who can follow along with your code. Our group
will always include students who are brand-new to programming and
coding, and we need our general-purpose materials to be comprehensible
to new users who may wish to use them as a starting point later. Many
(maybe most?) of our programs will include some code components, but
when it is possible to build in the interface, that's what we do.
Because we need for newbies to comprehend things, code components should
be obsessively commented.

\section{Record as much as you can}\label{record-as-much-as-you-can}

Though we are committed to documenting our hypotheses and analysis
pipelines before collecting data, it is inevitable that we will
sometimes think of a great way to test a hypothesis later. There is
nothing inherently wrong with data-driven analysis, so long as we are
honest about what we predicted before seeing outcomes, so we should try
to facilitate getting as much out of each precious data collection
opportunity as possible. Programmers must of course ensure that the
dependent variables needed for planned analyses are recorded properly,
but they should also consider whether other elements of the response may
also be conveniently collected. Examples:

\begin{enumerate}
\def\labelenumi{\arabic{enumi}.}
\item
  Your main interest may be response accuracies, but that should not
  prevent you from also collecting response speeds (and checking that
  you know how your program records them)
\item
  Your main dependent variable might be whether a whole sequential
  response is 100\% correct or not, but that should not prevent you from
  collecting the components of the response (e.g., that the participant
  responded 8675309, not just that they got a list ``right'' or
  ``wrong'').
\end{enumerate}

There could be good reasons to aurally record spoken responses, to have
response speeds for individual elements in a list response, for elements
I've not yet imagined. You should record as much as you ethically can,
and try to think in advance of the ways in which various response
elements would work together to converge on a hypothesis (and include
those in your preregistration when possible). If you want an element
later and did not record it, the only solution is to collect data again,
which costs time and (sometimes) money.

\section{There are always exceptions}\label{there-are-always-exceptions}

There will sometimes be projects for which we don't follow these
principles. Sometimes, we will have a good reason for wanting
participants to give natural, vocal responses, which currently cannot be
evaluated in real-time by a computer. There are bound to be other
exceptions. Expedience is \textbf{not} a good reason though. If you have
no idea how to implement something, ask for the support of the group.

\chapter{Experimental protocols}\label{experimental-protocols}

The experimental protocol is like a recipe for running your experiment.
Each experimenter responsible for a discrete data-collection initiative
must write the script for completing a run. This script should be
sufficiently thorough that a trust-worthy, non-lab-member psychologist
could run it correctly from the script alone. Imagine that you
experience an emergency shortly before your participant is scheduled to
arrive. Your protocol should be detailed enough that you could run into
the Tower building, find another Year 3 or PG student, give them your
protocol, and they could do everything from setting up the lab to saving
the data.

(Note, we would \textbf{never} actually do that. Of course non-lab
members cannot do our lab work - see data security. But nonetheless your
protocol should be detailed enough to allow it.)

\section{Writing your protocol}\label{writing-your-protocol}

The PI will direct you to an appropriate example protocol to work from.
Where possible, you may copy from the existing protocol. There no point
using your creativity to find new ways to describe instituting similar
set-ups. Familiarity also helps other lab members know when procedures
were the same and when they differed.

Writing a protocol is an exercise in theory-of-mind. You need to think
carefully about what someone else does not know about your procedure,
and address each point you can think of explicitly.

Our paradigms will typically include the following sections.

\subsection{Setting up}\label{setting-up}

Your protocol should start with whatever must be done \emph{before} the
first participant arrives. As a general rule, set-up should be complete
10 minutes before the participant is expected. Assume this, and work
backwards for determining how early setting up the lab for your
procedure must begin.

Set-up will typically include rebooting the computer(s), applying or
checking any settings (e.g., for volume, screen refreshing, color
temperature, etc.) specific to your experiment, and arranging the
workspace. Include every detail that needs to be applied or checked for
your experiment.

\subsection{Greeting and consent}\label{greeting-and-consent}

How will your participant find the lab? Will s/he be able to reach it on
their own, or do you need to meet them outside? You want the participant
to feel welcomed and not stressed about getting there, so plan to do
what it takes to ensure they know where to go.

Remember that your participant might never have been in a lab before.
They might not know what they may touch, where they may sit, etc.
Instruct them on everything - where to sit down, where to place their
things, etc. If they bring food or drink in, they may not have it during
the session, but you may be able to keep it somewhere safe for them
while they are working.

After the participant is settled, the first thing they must do is
consent to take part in the study. Consider how you will present consent
to them, and how you will emphasize the main points of the consent
document.

\subsection{Instructions and practice}\label{instructions-and-practice}

All of our experiments should be designed to guide participants through
the task, so that after an initial practice period the experimenter is
confident that the participant understands what to do. The best means
for doing this will vary dependning on your project. Here are some
general principles that we have found to be effective:

Do not plan to have your participants read through instructions on their
own. Of course they \emph{can} read, but always keep in mind that many
participants want to get in and out of the lab as quickly as possible.
They will assume that they can just figure it out, and click through
instructions like it's their iPhone legal agreement update. They will
get to a trial, and call you over because they do not know when to
respond or what button to press.

Either use the on-screen instructions to remind yourself of how you will
aurally explain the instructions to the participant, or implement a
read-instruction system where the participant cannot advance past a
screen on their own. One way to do this is to end the instructions with
something like ``Now, get the researcher's attention before
continuing'', without telling participants how to get off the screen
(and choosing a non-obvious exit key that the experimenter knows as the
only way to advance). This way, you have the opportunity to check that
the participant understands what they are about to do before they begin.

Include practice trials, and design them to be representative of
experimental trials, but easier. Over-represent the easiest conditions
in your practice block so that it will be really obvious to participants
what to do. Consider including an accuracy criterion for advancing to
the experimental trials, where the participant has to repeat the
practice block until they have proven they know how to do the task.
(Note: this is more advisable for some paradigms or components than
others, and only possible to implement for conditions where you expect
very high accuracy.)

Explain in your protocol how to guide participants through the
instructions and practice. After you have run some pilot participants,
you may want to update this section after seeing what they find
difficult.

\subsection{Monitoring or on-call}\label{monitoring-or-on-call}

For most of our studies, participants complete the experimental trials
independently after passing the practice block. There may be elements of
participants behavior that the researcher must monitor or record. If so,
what are these? How do you make recordings? What decisions will the
researcher need to make? Detail these instructions carefully so that
another could reproduce exactly what you do to monitor the participant.

If the researcher is simply ``on-call'', waiting for the participant to
finish, what is the researcher allowed to do? Is it fine to do some
quiet work during this time? Use the protocol to let the researcher know
that.

For what reasons is the participant likely to need assistance during
this time? Is there a new block of trials with different instructions?
Are there breaks? How often? Include these details in your protocol so
that the researcher knows what to expect. Participants might ask about
these details too.

\subsection{Saving and break-down}\label{saving-and-break-down}

When the participant is finished, the researcher should thank them,
debrief them (and pay them, if they are working for pay). The researcher
should escort the participant out of the secure lab area or building
(for the participant's convenience as well as to comply with safety and
security rules). What else needs to happen at the end of a run? Is there
anything to be recorded? How should the researcher save the new data?
After the final participant of the day, how should the lab be shut down?
If the participant is getting EMS credits, how does one apply those?

\subsection{Exceptions and unsual
events}\label{exceptions-and-unsual-events}

These typically include what to do if a participant decides to withdraw
consent. Participants may withdraw for any reason. They may request to
have their data deleted. In case they do, you should detail how one
would find the new data, and all the steps needed to show them it has
been deleted.

If the participant is taking part for money and withdraws consent, it is
only necessary to pay them for the amount of time they spent. Deciding
how to pro-rate participant can be sensitive. A good general rule is to
round up to the nearest quarter hour and apply the advertized hourly
rate.

If the participant is taking part for EMS credit, then how to proceed
depends on the reason they give. Do ask why they are withdrawing
consent, and delete their data if requested. Depending on the reason for
withdrawing, their credit might be pro-rated similarly to applying
payment. This should always be discussed with the PI.

\section{Testing your protocol}\label{testing-your-protocol}

New protocols must always be tested, and perhaps revised, before
beginning your study. If your experiment is part of an existing series,
you may be authorized to skip testing because subsequent experiments are
likely to be very similar to others in the series. But if your protocol
is for a new project, or you are new to the lab, you must take it
through the following tests.

First, run through it yourself. Try not to bring any unwritten knowledge
to interpreting your protocol. See if there is anything you need to add.
You should always do a complete run of your experiment too, to check
that it really does what you expect it do, stops when you think it will
stop, etc.

Then get another lab member to go through the protocol, and to actually
perform the set-up as the protocol instructs. If the lab member cannot
do it correctly based on the protocol instructions, then revise the
protocol based on their feedback and try again. The lab member may also
have suggestions for the remainder of the protocol for you to implement.
Run the lab member through your experiment if you have not already
pilot-tested it within the lab. The lab member may have suggestions for
tweaking your instructions, or may help find uncaught errors or bugs.

Once another lab member believes the protocol is fairly complete and the
experiment runs as intended, the protocol and experiment should be shown
to the PI. (Note: Depending on how much support the researcher needed to
program the experiment, the PI may already have seen the experiment
itself by this point.) If the PI finds it complete, then she will
authorize you to schedule a naive particpant for a time that she (or
another senior lab member) can observe. The observing lab member will be
present for the set-up, at least the instructions/practice if not the
whole run, and break-down to see how the protocol worked and how closely
the researcher followed it. After this exercise, the observing lab
member and researcher will discuss how the run went, whether any changes
need to be made. If no changes need to be made, then the run may be
considered the first participant in the study. This participant number
may need to be marked for exclusion if major changes are requested. This
decision should be explicit, and recorded in the notes the researcher
keeps about participant runs.

\section{Clearance to begin}\label{clearance-to-begin}

Once the researcher has completed a supervised run, and the supervising
lab member finds the run successful, with no changes to implement, then
the researcher is cleared to begin recruiting the necessary number of
participants.

\chapter{Dealing with the unexpected during data collection with adult,
student
participants}\label{dealing-with-the-unexpected-during-data-collection-with-adult-student-participants}

We design experimental materials to be easily administered and as
fail-safe as possible. However, that doesn't mean that nothing can go
wrong during your data collection. Here is a description of how
participant running usually goes, and then anecdotal exceptions that
have occurred to us more than once.

\section{The slow participant}\label{the-slow-participant}

Most student participants are not new to taking part in research, and
understand the drill. They want to get in and out of a session as
quickly as possible. This is so typical that we must account for it in
the design of experiments and ensure that participants learn that they
will not benefit from random responding. However, some participants take
much longer than is typical to respond. Sometimes, we never learn why,
we simply see that their session takes a long time and their response
speeds are much slower than average. Sometimes, these participants wish
to discuss how they are doing the task in detail with the experimenter.
If a participant wants to discuss the experiment with you, always engage
with them as far as possible (up to explaining the hypothesis, which you
can do after they finish).

\section{The inattentive participant}\label{the-inattentive-participant}

Remember how most student participants have learned that it is in their
best interest to finish as quickly as possible? Some will try to speed
everything along: they will grab the mouse and start trying to enter
information without knowing what they are doing, will start mashing
SPACE, ENTER and random keys to get the study going. Data from this
behavior would not signify anything. This is the reason our policy is
always to personally explain instructions and have a supervised practice
session. Practice sessions should be designed so that the experimenter
is certain the participant understands the task, and that the
participant is not inadvertently rewarded for speeding through examples
and instructions.

Sometimes we cannot ensure that a participant will not choose to
randomly respond after they are left on their own. Our tasks are often
hard, and it would not always be reasonable to impose an accuracy
criterion (though sometimes it is). Our strategy is to ensure that they
know how to do the task, not to set it up in such a way that
non-responding would allow them to finish earlier, and to build in some
trials that are meant to detect insincere attempts. Usually, this means
including some very easy condition. Even if these trials are not meant
to be part of the main design, they can be used to check that the
participant understood what to do and was not responding randomly.

\section{Sleepy, sick, or inebriated
participants}\label{sleepy-sick-or-inebriated-participants}

Student life is difficult, and sometimes you will greet a participant
who isn't in fit condition to be working on anything. The PI has seen
participants who came over directly from the pub, who were so ill that
they ought to have been in bed, or who were so tired they could hardly
sit up. These participants are trying too hard to meet their
requirements. If you encounter a participant in this state, the best
option is to offer to reschedule them. They will not learn anything from
trying to take part.

\section{The participant with a lot of
questions}\label{the-participant-with-a-lot-of-questions}

Occasionally, a participant will want to know a lot of detail about how
the study works or what will be done with the data. Intellectual
curiousity is always to be encouraged. Engage with questions as best you
can, but if you do not feel up to answering all of them, refer the
participant to the PI or another senior, responsible researcher.

Often these participants want to see their own data. Our general policy
is only to release anonymized data sets. Note down this participant's
email address so that we can alert him or her when the anonymized data
are available. In some cases, we may be okay with giving the participant
their own data set as well. The reason that we generally do not do this
is that in our experimental tasks, usually there is little (if anything)
to be learned from one person's performance. There is nothing diagnostic
we would be able to say about certain task scores, and there is some
worry that a participant might try to read too much into their
performance. Giving access to the anonymized data would be sufficient to
let them have a go at data analysis and satisfy their curiosity about
the results.

\bibliography{book.bib,packages.bib}


\end{document}
